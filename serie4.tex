\documentclass{article}

\usepackage[utf8]{inputenc}
\usepackage{amsmath}
\usepackage{mathtools}
\usepackage{enumerate}
\usepackage{listings}
\usepackage{color}

\definecolor{dkgreen}{rgb}{0,0.6,0}
\definecolor{gray}{rgb}{0.5,0.5,0.5}
\definecolor{mauve}{rgb}{0.58,0,0.82}

\lstset{frame=tb,
  language=Python,
  aboveskip=3mm,
  belowskip=3mm,
  showstringspaces=false,
  columns=flexible,
  basicstyle={\small\ttfamily},
  numbers=none,
  numberstyle=\tiny\color{gray},
  keywordstyle=\color{blue},
  commentstyle=\color{dkgreen},
  stringstyle=\color{mauve},
  breaklines=true,
  breakatwhitespace=true,
  tabsize=3
}
\newcounter{problem}
\newcounter{solution}

\newcommand\Problem[1]{%
  \stepcounter{problem}%
  \textbf{Problema #1.}~%
  \setcounter{solution}{0}%
}

\newcommand\TheSolution{%
  \textbf{Solución:}\\%
}

\newcommand\ASolution{%
  \stepcounter{solution}%
  \textbf{Solución \thesolution:}\\%
}
\setlength{\parindent}{0in}
\setlength{\parskip}{1em}

\title{Cuarta serie de ejercicios de Álgebra Moderna}
\date{Mayo 2018}
\author{Akiyuki Shinbou}
\begin{document}

\maketitle


\Problem{1} Encuentre el orden de las siguientes permutaciones. \begin{enumerate}[a.]
  \item $(14)$
  \item $(147)$
  \item $(14762)$
  \item $(a_{1}a_{2} \cdots a_k)$
\end{enumerate}

\TheSolution{} \begin{enumerate}[a.]
  \item $2$
  \item $3$
  \item $5$
  \item $k$
\end{enumerate}

\Problem{8} ¿Cual es el orden maximo de cualquier elemento en $A_{10}$?

\TheSolution{} Podemos escribir cada elemento de $A_{10}$ como un producto de
ciclos disjuntos, y el orden del elemento es el producto de esos ciclos.
Encontramos que el orden mayor que podemos obtener multiplicando estos ciclos
y seguir teniendo una permutación par es un par de permutaciones de ordenes 7
y 3, con orden 21, por ejemplo $(1234567)(89 10)$

\Problem{15} Sea $\alpha$ y $\beta$ en $S_n$. Demuestra que $\alpha\beta$ es
par si y solo si $\alpha$ y $\beta$ son ambos pares o ambo impares.

\TheSolution{} Podemos expresar $\alpha$ como un producto de $n$ 2-ciclos, y
$\beta$ un producto de $m$ 2-ciclos. Así, $\alpha\beta$ es un producto de
$m + n$ 2-ciclos, y $m + n$ solo es par si $m$ y $n$ son ambos pares o
impares.

\Problem{22} Sea $\alpha$ y $\beta$ en $S_n$. Demuestra que $\alpha^{-1}\beta
^{-1}\alpha\beta$ es una permutación par.

\TheSolution{} Podemos expresar $\alpha$ como un producto de $n$ 2-ciclos, y
$\beta$ un producto de $m$ 2-ciclos. Así, $\\alpha^{-1}\beta^{-1}alpha\beta$
es un producto de$m + n + m + n = 2(m + n)$, un numero par.

\Problem{29} Encuentra tres elementos $\sigma$ en $S_9$ con la propiedad de
que $\sigma^3 = (157)(283)(469)$

\TheSolution{} Observamos que al elevar el 9-ciclo $(a_1 a_2 a_3 a_4 a_5 a_6
a_7 a_8 a_9)$ al cubo obtenemos $(a_1 a_4 a_7)(a_2 a_5 a_8)(a_3 a_6 a_9)$.
Observando que $(157)(283)(469) = (157)(469)(283) = (283)(157)(469)$, llegamos
a los 9 ciclos $(124586739)$, $(142568793)$ y $(214856379)$.

\Problem{36} En $S_4$, encuentra un subgrupo ciclico de orden 4 y un subgrupo
no ciclico de orden 4.

\TheSolution{} $(1234)$ es un ciclo de orden cuatro, $\langle(1234)\rangle$ es
un subgrupo ciclico de orden 4. Un subgrupo no ciclico seria un conjunto con
solo permutaciones disjuntas, por ejemplo $\{ (1), (12), (34), (12)(34)\}$.

\Problem{43} Demuestra que $A_5$ tiene 24 elementos de orden 5, 20 elementos
de orden 3 y 15 elementos de orden 2.

\TheSolution{} Podemos descomponer todos los elementos de $A_5$ en 5-ciclos,
3-ciclos, o un producto de 2-ciclos disjuntos. Para los elementos de orden 5,
hay $5! / 5 = 24$ ciclos de la forma $(abcde)$. Hay $(5 * 4 * 3) / 3 = 20$
elementos de la forma $(abc)$. Para el caso de los elementos de orden 2,
encontramos los elementos de la forma $(ab)(cd)$, que son $(5 * 4 * 3 * 2)/8
= 15$, donde dividimos entre 8 porque existen 8 formas de escribir el mismo
par de 2-ciclos.

\Problem{50} Utiliza el esquema de verificación de digitos de Verhoeff basado
en $D_5$ para agregar un digito de verificación a 45723.

\TheSolution{} $\sigma(4) * \sigma^{2}(5) * \sigma^{3}(7) * \sigma^{4}(2) *
\sigma^{4}(3) = 2 * 9 * 5 * 5 * 3 = 5$. Necesitamos agregar el digito 5 para
que la suma se vuelva 0.

\Problem{57} ¿Por qué el hecho de que ordenes de los elementos de $A_4$ sean
1, 2 y 3 implica que $|Z(A_4)| = 1$?

\end{document}
