\documentclass{article}

\usepackage[utf8]{inputenc}
\usepackage{amsmath}
\usepackage{mathtools}
\usepackage{enumerate}
\usepackage{listings}
\usepackage{color}

\definecolor{dkgreen}{rgb}{0,0.6,0}
\definecolor{gray}{rgb}{0.5,0.5,0.5}
\definecolor{mauve}{rgb}{0.58,0,0.82}

\lstset{frame=tb,
  language=Python,
  aboveskip=3mm,
  belowskip=3mm,
  showstringspaces=false,
  columns=flexible,
  basicstyle={\small\ttfamily},
  numbers=none,
  numberstyle=\tiny\color{gray},
  keywordstyle=\color{blue},
  commentstyle=\color{dkgreen},
  stringstyle=\color{mauve},
  breaklines=true,
  breakatwhitespace=true,
  tabsize=3
}
\newcounter{problem}
\newcounter{solution}

\newcommand\Problem[1]{%
  \stepcounter{problem}%
  \textbf{Problema #1.}~%
  \setcounter{solution}{0}%
}

\newcommand\TheSolution{%
  \textbf{Solución:}\\%
}

\newcommand\ASolution{%
  \stepcounter{solution}%
  \textbf{Solución \thesolution:}\\%
}
\setlength{\parindent}{0in}
\setlength{\parskip}{1em}

\title{Cuarta serie de ejercicios de Álgebra Moderna}
\date{Mayo 2018}
\author{Akiyuki Shinbou}
\begin{document}

\maketitle


\Problem{1} Encuentre el orden de las siguientes permutaciones.
\begin{enumerate}[a.]
  \item $(14)$
  \item $(147)$
  \item $(14762)$
  \item $(a_{1}a_{2} \cdots a_k)$
\end{enumerate}

\TheSolution{} \begin{enumerate}[a.]
  \item $2$
  \item $3$
  \item $5$
  \item $k$
\end{enumerate}

\Problem{2} Escribe cada una de las siguientes permutaciones como un producto
de ciclos disjuntos.
\begin{enumerate}[a.]
  \item  (1235){(413)}
  \item (13256){(23)}{(46512)}
  \item (12){(13)}{(23)}{(142)}
\end{enumerate}

\TheSolution{}
\begin{enumerate}[a.]
\item(124){(35)}
\item (124){(35)}{(6)}
\item (1342)
\end{enumerate}

\Problem{3} ¿Cuál es el orden de las siguientes permutaciones?

\begin{enumerate}[a.]
\item $(124)(357)$
\item $(124)(3567)$
\item $(124)(35)$
\item $(124)(357869)$
\item $(1235)(24567)$
\item $(345)(245)$
\end{enumerate}

\TheSolution{}
\begin{enumerate}[a.]
  \item 3
  \item 12
  \item 6
  \item 6
  \item 12
  \item 2
\end{enumerate}

\Problem{4} ¿Cuál es el orden de cada una de las siguiente permutaciones?

\begin{enumerate}[a.]
  \item 
    \begin{bmatrix} 
      1 & 2 & 3 & 4 & 5 & 6 \\
      2 & 1 & 5 & 4 & 6 & 3
    \end{bmatrix}
  \item
    \begin{bmatrix}
      1 & 2 & 3 & 4 & 5 & 6 & 7 \\
      7 & 6 & 1 & 2 & 3 & 4 & 5
    \end{bmatrix}
\end{enumerate}

\TheSolution{}
\begin{enumerate}[a.]
  \item 6
  \item 12
\end{enumerate}

\Problem{5} ¿Cual es el orden del producto de un par de ciclos disjuntos de
longitud 4 y 6?

\TheSolution{}
\begin{align*}
  \textrm{Orden} &= lcm(4, 6)\\
  \textrm{Orden} &= 12
\end{align*}

\Problem{6} Demuestra que $A_6$ contiene un elemento de orden 15.

\TheSolution{}
Ya que el orden es el minimo comun multipl del largo de dos ciclos,
es suficiente encontrar una permutacion de 2-ciclos de largo 3
y una permutacion de 2-ciclos de largo 5 ya que $lcm(3, 5) = 15$.

Permutacion de largo 3  $(12) (23) (31)$

Permutacion de largo 5: $(45) (56) (67) (78) (84)$

\Problem{8} ¿Cual es el orden maximo de cualquier elemento en $A_{10}$?

\TheSolution{} Podemos escribir cada elemento de $A_{10}$ como un producto de
ciclos disjuntos, y el orden del elemento es el producto de esos ciclos.
Encontramos que el orden mayor que podemos obtener multiplicando estos ciclos
y seguir teniendo una permutación par es un par de permutaciones de ordenes 7
y 3, con orden 21, por ejemplo $(1234567)(89 10)$

\Problem{9} Determina cuales de las siguientes permutaciones son pares o
impares.
\begin{enumerate}[a.]
	\item (135)
  \item (1356)
  \item (13567)
  \item (12) (134) (152)
  \item (1243) (3521)
\end{enumerate}

\TheSolution{}
Ya que podemos expresar toda permutación como $k$ 2-ciclos, y el número de
2-ciclos por el que está formado determina si es par o impar, los obtendríamos
como sigue:

\[
  (135)=(13)(15)
\]

entonces este es un ciclo par, ya que tiene DOS 2-ciclos. Para los siguientes
se omitirá esta obvia conclusión y se deja desarrollado en su forma de $k$
2-ciclos.

\[{(1356)} = {(13)}{(15)}{(16)}\]
\[(13567) = (13)(15)(16)(17)\]
\[(12)(134)(152) = (15)(234)=(15)(23)(24)\]
\[(1243)(3521) = (12)(14)(13)(35)(32)\]

\Problem{10} Demuestra que una función de un conjunto finito a si mismo es en
si uno a uno si y solo si es sobre. ¿Esto es verdad cuando S es infinito?

\TheSolution{}
Si $\phi : S \mapsto S$ es uno a uno. Entonces $S$ y $S'$ = $\{\phi
(x) | x \in S\}$ tienen el mismo orden si $\nexists$ $a,b \in S$ $\mid a \neq
b$ y $\phi (a) = \phi (b)$ y si se cumple la condici\'on anterior entonces 
$|S| = |S'|$ y con ello $\phi$ es sobre.

Si $\phi : S \mapsto S$ es sobre. Entonces $S$ y $S'$ son el mismo
conjunto y dados los hechos que $|S| = |S'|$ y que $\phi$ es sobre,
$\phi$ debe ser uno a uno.

No es verdad cuando S es infinito pues funciones que realicen un múltiplo de
un elemento, i.e $\alpha : \mathbb{Z} \mapsto \mathbb{Z}$ donde $\alpha (a) =
ka$, podrán ser uno a uno pero no son sobre.

\Problem{12} Si $\alpha$ es par, demuestra que $\alpha^{-1}$ es par. Si
$\alpha$ es impar, demuestra que $\alpha^{-1}$ es impar.

\TheSolution{}
Sea $\alpha$ una multiplicación de 2-ciclos (dado que se puede representar
cualquier permutaci\'on finita como una multiplicación de 2-ciclos). Y también
sea $\alpha = x_1,x_2,\dots,x_n$ para algun entero positivo donde $x_i$ es un
2-ciclo para toda $i=1,\dots,n$. Entonces se tiene:\\
		\begin{align*}
      \alpha^{-1} &= {(x_1,x_2,\dots,x_n)}^{-1}\\
			&= x_1^{-1}, x_2^{-1},\dots,x_n^{-1}\\
		\end{align*}
Desde que la inversa de cada 2-ciclos es si mismo, entonces $x_i^{-1} = x_i$
por cada $i=1,\dots,n$. Despu\'es, se puede notar que $\alpha^{-1}$ tiene el
mismo n\'umero de 2-ciclos que $\alpha$, entonces si $\alpha$ es par, entonces
$\alpha^{-1}$ es par tambi\'en, y lo mismo ocurre si $\alpha$ es impar,
$\alpha^{-1}$ es impar.

\Problem{13} Demuestra que el conjunto de permutaciones pares en $S_n$ forman
un subgrupo de $S_n$.

\TheSolution{}
Se quiere probar que $A_{n}$ es un subgrupo de $S_{n}$.
Para esto es necesario probar:

\begin{enumerate}
    \item $p, q \in A_{n} \rightarrow pq \in A_{n}$
    
    \item $e \in A_{n}$
    
    \item $p \in A_{n} \rightarrow p^{-1} \in A_{n}$
\end{enumerate}

El segundo punto es trivial y el primer y tercer punto se pueden juntar en uno
solo: $p q^{-1} \in A_{n}$ y es suficiente para demostrarlos.

Entonces digamos que $p = p_{1} p_{2} \ldots  p_{2k}$ y
$q = q_{1} q_{2} \ldots  q_{2j}$

$q^{-1} = q^{-1}_{2j} q^{-1}_{2j-1} /ldots q^{-1}_{1}$ por el teorema
calceta-zapato

Ahora, $p q^{-1} = p_{1} p_{2} /ldots p_{2k} q^{-1}_{2j} q^{-1}_{2j-1} /ldots q_{2}
q_{1}$ lo cual es un producto de un numero par de transposiciones.

$p \in S_{n}, q \in S_{n}, q^{-1} \in S_{n}$ y $S_{n}$ es cerrado asi que se
cumple lo que se queria probar, $p q^{-1} \in S_{n}$.

\Problem{15} Sea $\alpha$ y $\beta$ en $S_n$. Demuestra que $\alpha\beta$ es
par si y solo si $\alpha$ y $\beta$ son ambos pares o ambo impares.

\TheSolution{} Podemos expresar $\alpha$ como un producto de $n$ 2-ciclos, y
$\beta$ un producto de $m$ 2-ciclos. Así, $\alpha\beta$ es un producto de
$m + n$ 2-ciclos, y $m + n$ solo es par si $m$ y $n$ son ambos pares o
impares.

\Problem{16} Asocia una permutación par con el numero $+1$ y una permutación
impar con el numero $-1$. Has una analogia entre el resultado de multiplicar
dos permutaciones y el resultado de multiplicar sus numeros $1$ o $-1$
correspondientes.

\TheSolution{}
\begin{table}[htbp]
\begin{center}
\begin{tabular}{c|cc}
* & Odd & Even \\ \hline
Odd & Odd  & Even\\
Even & Even & Odd \\
\end{tabular}
\caption{Técnica pedagógica avanzada para mostrar el producto de permutaciones
pares e impares}
\end{center}
\end{table}

Recordemos nuestros buenos tiempos en la primaria, cuando para familiarizarnos
con el producto de enteros con o sin signo, nos pudieron haber mostrado esta
tablita:

\begin{table}[htbp]
\begin{center}
\begin{tabular}{c|cc}
* & + & - \\ \hline
+ & +  & -\\
- & - & + \\
\end{tabular}
\end{center}
\end{table}

\Problem{17} Sea 
$\alpha = \begin{bmatrix}
  1 & 2 & 3 & 4 & 5 & 6 \\
	2 & 1 & 3 & 5 & 4 & 6
\end{bmatrix}$ y 
$\beta = \begin{bmatrix}
  1 & 2 & 3 & 4 & 5 & 6 \\
  6 & 1 & 2 & 4 & 3 & 5
\end{bmatrix}$

Computa lo siguiente.
\begin{enumerate}[a.]
  \item $\alpha^{-1}$
  \item $\beta \alpha$
  \item $\alpha \beta$
\end{enumerate}

\TheSolution{}
\begin{enumerate}[a.]
  \item $\alpha^{-1} = (12)(54)$
  \item $\beta\alpha = 
    \begin{bmatrix}
      1 & 2 & 3 & 4 & 5 & 6 \\
			1 & 6 & 2 & 3 & 4 & 5
    \end{bmatrix}$
  \item $\alpha\beta = 
    \begin{bmatrix}
      1 & 2 & 3 & 4 & 5 & 6 \\
			6 & 2 & 1 & 5 & 3 & 4
    \end{bmatrix}$
\end{enumerate} 
\Problem{19} Demuestra que si $H$ es un subgrupo de $S_n$, entonces todo
numero de $H$ es una permutación par o exactamente la mitad de los elementos
es par.

\TheSolution{}
Sea $H$ un subgrupo de $S_n$. Si $H$ no contiene ninguna permutaci\'on impar,
entonces H solo contiene permutaciones par.

Por otro lado, si $H$ contiene permutaciones impares, sea $\alpha \in H$ una
permutación impar, y consideremos la siguiente función $f: H \rightarrow
H$, donde $f(h) = \alpha \cdot h$, donde si $h$ (el cual pertenece a $H$) es
par, $f(h) = \alpha h$ es impar (par$\cdot$impar$=$impar). Desde que $h$ y
$\alpha$ estan en $H$, entonces $f(h)=\alpha h$ esta en $H$ también. Esto
conlleva a que $f$ envia permutaciones pares en $H$ a permutaciones impares
en $H$.

Tambi\'en se puede notar que la funci\'on $f$ es inyectiva: Suponga que
$f(i) = f(j)$. Esto es $\alpha i = \alpha j$, y las $\alpha$s se pueden
cancelar multiplicando $\alpha^{-1}$ por la izquierda, y esto deja que $i=j$,
por lo que $f$ es inyectiva.

También se puede ver que $f$ es sobreyectiva. Sea $i$ una permutación
impar. Entonces $\alpha^{-1}i$ es una permutación par (porque impar$\cdot$
impar$=$par), y esta en $H$ porque $\alpha$ (consecuentemente, también $\alpha
^{-1}$) e $i$ están en $H$. Consecuentemente $\alpha^{-1}i \in H$. También
se nota que $f(\alpha^{-1}i) = \alpha\alpha^{-1}i = i$, y esto sigue a que $f$
es sobreyectiva.

Esto demuestra a que hay una función inyectiva y sobreyectiva $f$ par
$\rightarrow $ impar, entonces $|$par$|=|$impar$|$. Entonces la mitad de las
permutaciones de $H$ son pares y la otra mitad son impares.

\Problem{20} Computa el orden de cada miembro de $A_4$. ¿Cual es la relación
aritmetica entre estos ordenes y el orden de $A_4$?

\TheSolution{}
$|e| = 1$, $|(12)(34)| = 2$, $|(13)(24)| = 2$, $|(14)(23)| = 2$, $|(123)| = 
3$, $|(132)| = 3$, $|(124)| = 3$, $|(142)| = 3$, $|(134)| = 3$, $|(143)|$,
$|(234)| = 3$, $|(243)| = 3$

El orden de $A_4$ es 12, y todos los ordenes de sus elementos dividen a 12.

\Problem{22} Sea $\alpha$ y $\beta$ en $S_n$. Demuestra que $\alpha^{-1}\beta
^{-1}\alpha\beta$ es una permutación par.

\TheSolution{} Podemos expresar $\alpha$ como un producto de $n$ 2-ciclos, y
$\beta$ un producto de $m$ 2-ciclos. Así, $\\alpha^{-1}\beta^{-1}alpha\beta$
es un producto de$m + n + m + n = 2(m + n)$, un numero par.

\Problem{24} ¿Cuantos elementos de orden 5 hay en $S_7$?

\TheSolution{}
Aquellas permutaciones son las del tipo:
\[
  (5)(1)(1)	
\]
Se eligen 5 de los 7 n\'umeros, lo cual es:
\[
  \left(
    \begin{array}{c}
      7 \\
      5
    \end{array}
  \right)
  = 21
\]
Considerar las formas diferentes de acomodar dichos n\'umeros eliminando repeticiones es:
\[
  \frac{5!}{5} = 24	
\]
Por lo tanto el resultado es $21 \times 24 = 504$

\Problem{26} Demuestra que (1234) no es un producto de 3-ciclos.

\TheSolution{}
Sea $y=(1234)$, esto también es igual a $(1,2)(1,3)(1,4)$, el cual es impar.

Entonces, sea $y = y_1y_2\dots y_m$, y sea $y_i = (a_{1i}a_{2i}a_{3i})$ (todas
las $y$'s son 3-ciclos)

Pero cada $y_i$ es par ya que $y_i$ tambi\'en se representar como $y_i = (a_
{1i}a_{2i})(a_{1i}a_{3i})$,

Entonces $y$ es igual al producto de 2-ciclos pares, pero la hip\'otesis dice
que que $(1234)$ es un producto de 3-ciclos y eso genera una contradicci\'on.

\Problem{27} Sea $\beta \in S_7$ y supon que $\beta^4 = (2143567)$. Encuentre
$\beta$.

\TheSolution{}
Sabemos que ${(\beta^4)}^7 = \beta^{28} = e$, por lo tanto, $|\beta|$ divide a
28. De entrada sabemos que $|\beta| \neq 1$ Si $|\beta| = 14$, entonces la
representación en ciclos disjuntos de $\beta$ necesitaria al menos un ciclo de
orden 7 y uno de orden 2, lo cual requeriria 9 simbolos, donde solo tenemos 7
disponibles. Un argumento similar sirve para descartar que $|\beta| = 28$.
Por eliminación, tenemos que $|\beta| = 7$, de donde obtenemos que $\beta^8 =
\beta^4 \beta^4 = \beta = (2457136)$




\Problem{29} Encuentra tres elementos $\sigma$ en $S_9$ con la propiedad de
que $\sigma^3 = (157)(283)(469)$

\TheSolution{} Observamos que al elevar el 9-ciclo $(a_1 a_2 a_3 a_4 a_5 a_6
a_7 a_8 a_9)$ al cubo obtenemos $(a_1 a_4 a_7)(a_2 a_5 a_8)(a_3 a_6 a_9)$.
Observando que $(157)(283)(469) = (157)(469)(283) = (283)(157)(469)$, llegamos
a los 9 ciclos $(124586739)$, $(142568793)$ y $(214856379)$.

\Problem{30} ¿Qué ciclo es ${(a_{1}a_{2}\cdots a_{n})}^{-1}$?

\TheSolution{}
Dado que la identidad la podemos expresar como

\[(a_{1})(a_{2})(a_{3})\cdots (a_{n})\]

en el conjunto de permutaciones, entonces, bajo el proceso de operar de
derecha a izquierda, buscamos que el producto de tal permutación por su
inversa nos regrese, primero, $(a_{1})$, y así sucesivamente. De tal manera
que la única manera sería 

\[(a_{n}a_{n-1}a_{n-2}\cdot\cdot\cdot a_{1})\]

y esa sería la inversa.

\Problem{31} Sea $G$ un grupo de permutaciones en el conjunto $X$. Sea $a \in
X$ y definimos $stab(a) = \{\alpha \in G | \alpha(a) = a\}$. Llamamos a
$stab(a)$ el estabilizador de a en $G$ (ya que consiste en todos los miembros
de G que mantienen a $a$ fija). Demuestra que $stab(a)$ es un subgrupo de G.

\TheSolution{}
Sabemos que $e \in $ $stab(a)$, por lo que $stab(a)$ $\neq \emptyset$. Si
usamos el teorema 3.1 donde para ver que $H$ es un subconjunto de $G$ entonces
hay que demostrar que $bc^{-1}$ para cualquier $b,c \in$ $stab(a)$. Sabemos
que si $e \in$ $stab(a)$ entonces $c^{-1}$ debe estar en $stab(a)$ pues $cc^
{-1} = e$. Por lo tanto $bc^{-1}(a)$ = $b(a)$ = a, así que $stab(a)$ es un
subgrupo de $G$.

\Problem{33} Sea $alpha = (1, 3, 5, 7, 9)(2, 4, 6)(8, 10)$. Si $\alpha$ es
5-ciclo, ¿Qué puedes decir acerca de $m$?

\TheSolution{}
\[
\begin{split}
  \alpha   &= (1, 3, 5, 7, 9)(2, 4, 6)(8, 10) \\
  |\alpha| &= lcm(5,3,2) = 30 \\
\end{split}
\]
  $\alpha^m $ es un 5-ciclo. 
\[
\begin{split}
  \alpha^{5m} &= e = \alpha^{30n} \text{ donde } n \in N \\
  5m &= 30n \\
  m &= 6k \text{ donde } k \in N \\
\end{split}
\]

$m$ es múltiplo de 6.

\Problem{36} En $S_4$, encuentra un subgrupo ciclico de orden 4 y un subgrupo
no ciclico de orden 4.

\TheSolution{} $(1234)$ es un ciclo de orden cuatro, $\langle(1234)\rangle$ es
un subgrupo ciclico de orden 4. Un subgrupo no ciclico seria un conjunto con
solo permutaciones disjuntas, por ejemplo $\{ (1), (12), (34), (12)(34)\}$.

\Problem{37} Supón que $\beta$ es un 10-ciclo. ¿Para cuales enteros $i$ entre
2 y 10 $\beta i$ tambien es un 10-ciclo?

\TheSolution{}
Al calcular $\beta ^{2} = (a_{1}a_{2}\cdot\cdot\cdot a_{10})\cdot (a_{1}a_{2}
\cdot\cdot\cdot a_{10})$  notamos que los si empezamos la permutación
resultado con algún $a_{k}$ tal que $k\in [1,10]$, será de la forma:
\[
  (a_{(k)\mod10}a_{(k+2)\mod10}\cdot\cdot\cdot a_{(k+8)\mod10})\cdot (a_{(k+1)
  \mod10}a_{(k+3)\mod10}\cdot\cdot\cdot a_{(k+9)\mod10})
\]

Es decir, que obtenemos dos 5-ciclos. Esto se debe a que el $2$ divide al
$10$. En general, para saber si nuestra $n$ para $\beta ^{n}$ hace que esta
última expresión sea un 10-ciclos, necesitamos que los múltiplos de $n$ módulo
10 nos generen los diez elementos. Por ejemplo, para $n=3$:

$3,6,9$. Sin necesidad de operarlo por el módulo 10.
$12\mod10=2,15\mod10=5,18\mod10=8$. Ahora tenemos otros tres elementos más que
son generados en el mismo ciclo. Lo equivalente para los siguientes tres, que
son $1,4,7$. Entonces, para $n=3$ obtenemos un 10-ciclo. Lo mismo para $7$ y
$9$. En general, es suficiente que $n$ y 10(o incluso cualquier ciclo) sean
primos relativos. Lo que implica que tampoco funciona, en este caso, para el 4
ni el 8.

\Problem{38} En $S_3$, encuentra elementos $\alpha$ y $\beta$ tales que $|
\alpha| = 2$, $|\beta| = 2$ y $|\alpha\beta| = 3$.

\TheSolution{}
$\alpha = (13)(2)$, $\beta = (12)(3)$, $\alpha\beta = (123)$.

\Problem{41} Demuestra que $S_n$ es no abeliano para toda $n \geq 3$.
Contra ejemplo:

Ya que $n \geq 3$, los 2-ciclos $a = (12)$ y $b = (13)$ estan en $S_{n}$.

\[ab = (12)(13) = (132)\]
\[ba = (13)(12) = (123)\]
\[(132) \neq (123)\]

Por lo tanto $S_{n}$ con $n \geq 3$ no es abeliano.


\Problem{43} Demuestra que $A_5$ tiene 24 elementos de orden 5, 20 elementos
de orden 3 y 15 elementos de orden 2.

\TheSolution{} Podemos descomponer todos los elementos de $A_5$ en 5-ciclos,
3-ciclos, o un producto de 2-ciclos disjuntos. Para los elementos de orden 5,
hay $5! / 5 = 24$ ciclos de la forma $(abcde)$. Hay $(5 * 4 * 3) / 3 = 20$
elementos de la forma $(abc)$. Para el caso de los elementos de orden 2,
encontramos los elementos de la forma $(ab)(cd)$, que son $(5 * 4 * 3 * 2)/8
= 15$, donde dividimos entre 8 porque existen 8 formas de escribir el mismo
par de 2-ciclos.

\Problem{47} Demuestra que cada elemento en $A_n$ para $n \geq$ puede ser
expresada como un 3-ciclo o un producto de tres ciclos.

Sea $\alpha \in A_n$ para $n \geq 3$

\[
\begin{split}
  \alpha           &= (a_1a_2)(a_3a_4)\dots(a_{m-1}a_m) \\
  (a_1a_2)(a_3a_4) &= (ab)(cd) o (ab)(bc) \\
\end{split}
\]
Si los 4 son distintos:
\[
\begin{split}
  \tab (a_1a_2) (a_3a_4) &= (a_1a_2a_3) (a_3a_1a_4)\\
  \text{Si uno de ellos es igual, por ejemplo} a_2 = a_3\\
  \tab (a_1a_2) (a_3a_4) &= (a_1a_3a_2)
\end{split}
\]

\Problem{50} Utiliza el esquema de verificación de digitos de Verhoeff basado
en $D_5$ para agregar un digito de verificación a 45723.

\TheSolution{} $\sigma(4) * \sigma^{2}(5) * \sigma^{3}(7) * \sigma^{4}(2) *
\sigma^{4}(3) = 2 * 9 * 5 * 5 * 3 = 5$. Necesitamos agregar el digito 5 para
que la suma se vuelva 0.

\Problem{57} ¿Por qué el hecho de que ordenes de los elementos de $A_4$ sean
1, 2 y 3 implica que $|Z(A_4)| = 1$?

Tenemos los elementos de $A_4$ y sus respectivos ordenes:
$|e| = 1$, $|(12)(34)| = 2$, $|(13)(24)| = 2$, $|(14)(23)| = 2$, $|(123)| = 
3$, $|(132)| = 3$, $|(124)| = 3$, $|(142)| = 3$, $|(134)| = 3$, $|(143)|$,
$|(234)| = 3$, $|(243)| = 3$

Podemos ver que de aqui el unico elemento que conmuta con los demas es $e$,
por lo que $|Z(A_4)| = 1$.

\end{document}
