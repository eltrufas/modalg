\documentclass{article}

\usepackage[utf8]{inputenc}
\usepackage{amsmath}
\usepackage{mathtools}
\usepackage{enumerate}
\usepackage{listings}
\usepackage{color}

\definecolor{dkgreen}{rgb}{0,0.6,0}
\definecolor{gray}{rgb}{0.5,0.5,0.5}
\definecolor{mauve}{rgb}{0.58,0,0.82}

\lstset{frame=tb,
  language=Python,
  aboveskip=3mm,
  belowskip=3mm,
  showstringspaces=false,
  columns=flexible,
  basicstyle={\small\ttfamily},
  numbers=none,
  numberstyle=\tiny\color{gray},
  keywordstyle=\color{blue},
  commentstyle=\color{dkgreen},
  stringstyle=\color{mauve},
  breaklines=true,
  breakatwhitespace=true,
  tabsize=3
}
\newcounter{problem}
\newcounter{solution}

\newcommand\Problem[1]{%
  \stepcounter{problem}%
  \textbf{Problema #1.}~%
  \setcounter{solution}{0}%
}

\newcommand\TheSolution{%
  \textbf{Solución:}\\%
}

\newcommand\ASolution{%
  \stepcounter{solution}%
  \textbf{Solución \thesolution:}\\%
}
\setlength{\parindent}{0in}
\setlength{\parskip}{1em}

\title{Quinta serie de ejercicios de Álgebra Moderna}
\author{Akiyuki Shinbou}
\date{Mayo 2018}

\begin{document}

\maketitle

\Problem{1} Encuentra un isomorfismo del grupo de los enteros bajo la suma al
grupo de los enteros pares bajo la suma.

\TheSolution{} $\phi(n) = 2n$. Es uno a uno porque $2n = 2m$ implica que $a =
b$. $\phi(m + n) = 2(m + n) = 2m + 2n$, asi que la operacion del grupo se
conserva.

\Problem{8} Demuestra que el mappeo $a \rightarrow \log_{10}a$ es un
isomorfismo de $R^+$ bajo la multiplicación a $R$ bajo la suma.

\TheSolution{} La definición del logaritmo asegura que el mappeo es uno a uno
sobre $R$, y las leyes de los logaritmos dicen que $\log_{10}(ab) = \log_{10}
a + \log_{10}b$, por lo que la operación se preserva.

\Problem{15} Si $G$ es un grupo, demuestra que $Aut(G)$ y $Inn(G)$ son grupos
[bajo la operación de composición de funciones].

\TheSolution{} Tomamos un elemento $\alpha \in Aut(G)$. Para probar que
$Aut(G)$ es un grupo, solo hace falta mostrar que $\alpha^{-1}$ preserva la
operación del $G$:
\[
\begin{split}
  \alpha^{-1}(xy)              &= \alpha^{-1}x\alpha^{-1}y \\
  \iff \alpha(\alpha^{-1}(xy)) &= \alpha(\alpha^{-1}x\alpha^{-1}y) \\
  \iff xy                      &= \alpha(\alpha^{-1}x)\alpha(\alpha^{-1}y) \\
                               &= xy
  \end{split}
\]

Por lo que $Aut(G)$ es un grupo.

\Problem{29} Sean $C$ los numeros complejos y

\[
  M = \Bigg\{
\begin{bmatrix}
  a & -b \\
  b & a
\end{bmatrix}
\Big| a, b \in R
\Bigg\}
\]

Demuestre que $C$ y $M$ son isomorficos bajo la suma y que $C*$ y $M*$, los
elemenos no cero de $M$, son isomorficos bajo la multiplicación.

\TheSolution{}
Tomamos el isomorfismo
\[
\phi(a + bi)
  =
\begin{bmatrix}
  a & -b \\
  b & a
\end{bmatrix}
\]

Revisamos la operación:

\[
\begin{split}
  \phi((a + bi) + (c + di)) &=
  \begin{bmatrix}
    a + c & -(b + d) \\
    b + d & a + c
  \end{bmatrix}
  = \\
  \begin{bmatrix}
    a & -b \\
    b & a
  \end{bmatrix}
  +
  \begin{bmatrix}
    c & -d \\
    d & c
  \end{bmatrix}
  &=
  \phi(a + bi) + \phi(c + di)
\end{split}
\]

Y para el otro caso:


\begin{align}
  \phi((a + bi)(c + di)) = \phi((ac - bd) + (ad + bc)i) = \\
  \begin{bmatrix}
    ac - bd & -(ad + bc) \\
    ac + bd & ac - bd
  \end{bmatrix}
  =
  \begin{bmatrix}
    a & -b \\
    b & a
  \end{bmatrix}
  \begin{bmatrix}
    c & -d \\
    d & c
  \end{bmatrix}
\end{align}


\Problem{33} Sea $G$ un grupo y sea $g \in G$. Si $z \in Z(G)$, muestra que
el automorfismo interno inducido por $g$ es el mismo que el automorfismo
inducido por $zg$ (esto es, que los mappeos $\phi_g$ y $\phi_{zg}$ son
iguales).

\TheSolution{} $\phi_g(x) = gxg^{-1}$ y $\phi_{zg}(x) = zgx{(zg)}^{-1} =
zgxz^{-1}g^{-1} = zz^{-1}gxg^{-1} = gxg^{-1}$ ya que $z \in Z(G)$.

\Problem{35} Supón que $g$ y $h$ inducen el mismo automorfismo interno de un
grupo $G$. Demuestra que $h^{-1}g \in Z(G)$.

\TheSolution{} Si $\phi_g = \phi_h$, entonces $gxg^{-1} = hxh^{-1}$ para toda
$x$. Entonces $x = h^{-1}gxg^{-1}h = h^{-1}gx{(h^{-1}g)}^{-1}$, lo que implica
que $h^{-1}g \in Z(G)$.

\Problem{36} Combina los resultados del ejercicio 33 y 35 en un solo teorema
``Si y solo si''.

\TheSolution{} $\phi_g = \phi_h$ si y solo si $h^{-1}g \in Z(G)$

\Problem{43} Demuestra que $Q^+$, el grupo de los numeros racionales positivos
bajo la multiplicación, es isomorfo a un subgrupo propio.

\TheSolution{} $\phi(x) = x^2$ es uno a uno, ya que si $a^2 = b^2$, $a = b$
para $a$ y $b$ en $Q^+$. La operación se preserva, ya que $\phi(ab) = (ab^2) =
a^2b^2 = \phi(a)\phi(b)$. Sin embargo, $\phi$, no es sobre, ya que no existe
numero racional cuyo cuadrado sea 2, asi que la imagen de $\phi$ forma un
subgrupo propio de $Q^+$.

\end{document}
