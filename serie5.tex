\documentclass{article}

\usepackage[utf8]{inputenc}
\usepackage{amsmath}
\usepackage{mathtools}
\usepackage{enumerate}
\usepackage{listings}
\usepackage{color}

\definecolor{dkgreen}{rgb}{0,0.6,0}
\definecolor{gray}{rgb}{0.5,0.5,0.5}
\definecolor{mauve}{rgb}{0.58,0,0.82}

\lstset{frame=tb,
  language=Python,
  aboveskip=3mm,
  belowskip=3mm,
  showstringspaces=false,
  columns=flexible,
  basicstyle={\small\ttfamily},
  numbers=none,
  numberstyle=\tiny\color{gray},
  keywordstyle=\color{blue},
  commentstyle=\color{dkgreen},
  stringstyle=\color{mauve},
  breaklines=true,
  breakatwhitespace=true,
  tabsize=3
}
\newcounter{problem}
\newcounter{solution}

\newcommand\Problem[1]{%
  \stepcounter{problem}%
  \textbf{Problema #1.}~%
  \setcounter{solution}{0}%
}

\newcommand\TheSolution{%
  \textbf{Solución:}\\%
}

\newcommand\ASolution{%
  \stepcounter{solution}%
  \textbf{Solución \thesolution:}\\%
}
\setlength{\parindent}{0in}
\setlength{\parskip}{1em}

\title{Quinta serie de ejercicios de Álgebra Moderna}
\author{Akiyuki Shinbou}
\date{Mayo 2018}

\begin{document}

\maketitle

\Problem{1} Encuentra un isomorfismo del grupo de los enteros bajo la suma al
grupo de los enteros pares bajo la suma.

\TheSolution{} $\phi(n) = 2n$. Es uno a uno porque $2n = 2m$ implica que $a =
b$. $\phi(m + n) = 2(m + n) = 2m + 2n$, asi que la operacion del grupo se
conserva.

\Problem{2} Encuentre $Aut(Z)$.

\TheSolution{}
Sea $\phi(x): x$. Es el caso más trivial. Supongamos que $\phi (x)=\phi (y)$.
Entonces $x=y$ directamente; con ello probamos que es uno-a-uno. Para encontrar
$x$ tal que $\phi (x)=y$, esto ocurre cuando $x=y$. Y al ser automorfismo claro
que se conserva la operación.

$$\phi (x+y)= x+y$$
$$\phi(x)+\phi(y)=x+y$$

Ya que $\phi :\Z \rightarrow \Z$, definimos con un $+$ la operación del grupo
en ambos lados sin que exista ambigüedad. 

\Problem{3} Sea $R^+$ el grupo de los reales positivos bajo la multiplicación.
Demuestre que el mappeo $\phi(x) = \sqrt{x}$ es un automorfismo de $R^+$.

\TheSolution{} $\phi$ es sobre porque está definido para todos los
reales positivos. Es uno a uno porque $\sqrt{a} = \sqrt{b}$ implica $a = b$.
Finalmente, $\phi(xy) = \sqrt{xy} = \sqrt{x}\sqrt{y} = \phi(x)\phi(y)$.

\Problem{4} Muestre que $U(8)$ no es isomorfo a $U(10)$.

\TheSolution{}
Note que $U(10)$ es cíclico, pero $U(8)$ no lo es. Sabemos que un grupo es
cíclico syss si un grupo isomorfo también es cíclico, por lo tanto, $U(8)$ no
es isomorfo a $U(10)$.

\Problem{5} Demuestre que $U(8)$ es isomorfo a $U(12)$.

\TheSolution{}
$\\U(8) = \{1,3,5,7\}$\\
		$U(12) = \{1,5,7,11\}$\\
		\\
    Primero se define la funci\'on $\phi$, y desde que $|U(8)| = |U(12)|$
    podemos definir $\phi$ como $\phi: U(8) \rightarrow U(12)$, donde:\\
    $\phi(1) = 1$\\
		$\phi(3) = 5$\\
		$\phi(5) = 7$\\
		$\phi(7) = 11$\\
		\\
    Con el paso anterior se puede observar que evidentemente la funci\'on es
    inyectiva y sobreyectiva.\\ Ahora se probara si se mantiene la
    multiplicaci\'on donde $\phi(ab) = \phi(a)\phi(b)$ para toda $a$ y $b$ en
    $U(8)$:\\
		$\phi(1\cdot n) = \phi(1)\phi(n) = 1\phi(n)$ para toda $n \in U(8)$.\\
		$\phi(3\cdot 5) = \phi(15) = \phi(7) = 11'= 5'7'= \phi(3)\phi(5)$.\\
		$\phi(3\cdot 7) = \phi(21) = \phi(5) = 7' = 5'11' = \phi(3)\phi(7)$.\\
		$\phi(5\cdot 7) = \phi(35) = \phi(3) = 5' = 7'11' = \phi(5)\phi(7)$.\\
		La multiplicaci\'on se mantiene. Y por lo tanto $U(8) \approx U(12)$

\Problem{6} Demuestra que la noción de un grupo es transitiva. Es decir, si
$G$, $H$ y $K$ son grupos y $G \approx H$ y $H \approx K$, entonces $G \approx
K$.

\TheSolution{}
Tomamos $\alpha$ como un isomorfismo de $G$ a $H$, y $\beta$ un isomorfismo de
$H$ a $K$. El teorema 0.7 nos asegura que $\beta\alpha$ es uno a uno y sobre.
Si tomamos $a, b \in G$, $(\beta\alpha)(ab) = \beta(\alpha(ab)) =
\beta(\alpha(a)\alpha(b)) = \beta(\alpha(a))\beta(\alpha(b)) = (\beta\alpha)
(a)(\beta\alpha)(b)$.

\Problem{8} Demuestra que el mappeo $a \rightarrow \log_{10}a$ es un
isomorfismo de $R^+$ bajo la multiplicación a $R$ bajo la suma.

\TheSolution{} La definición del logaritmo asegura que el mappeo es uno a uno
sobre $R$, y las leyes de los logaritmos dicen que $\log_{10}(ab) = \log_{10}
a + \log_{10}b$, por lo que la operación se preserva.

\Problem{9} En la notación del teorema 6.1, demuestra que $T_e$ es la identidad
y que ${(T_g)}^{-1} = T_g^{-1}$.

\TheSolution{}
El Teorema 6.1 dice que todo grupo es isomorfo a un grupo de permutaciones. En
esos términos, $T_{e}=ex=x$. Entonces es la identidad. Para conocer la inversa
basta con $T_{g}T_{h}=T_{gh}$$ $$T_{g}T_{g^{-1}}=T_{gg^{-1}}=e$

\Problem{10} Sea $G$ un grupo. Demuestra que el mappeo $\alpha(g) = g^{-1}$
para toda $g$ en $G$ es un automorfismo si y solo si $G$ es abeliano.

\TheSolution{}
Supóngamos que $\alpha$ es un automorfismo. Entonces $\alpha(ab) =
{(ab)}^{-1}$. Ademas, $\alpha(ab) = \alpha(a) \alpha(b) = a^{-1}b^{-1}$, por lo
que $a^{-1} b^{-1} = {(ab)}^{-1} = b^{-1}a^{-1}$, por lo que $G$ es abeliano.

Por otro lado, si $G$ es abeliano, tenemos que $\phi(ab) = {(ab)}^{-1} =
{(ba)}^{-1} = a^{-1}b^{-1} = \phi(a)\phi(b)$.

\Problem{11} Para automorfismos internos $\phi_g$, $\phi_h$ y $\phi_{gh}$,
demuestra que $\phi_g\phi_h = \phi_{gh}$.

\TheSolution{}
Para toda $x$ en el grupo, tenemos que $(\phi_g\phi_h)(x) = \phi_g(\phi_h(x)) =
\phi_g(hxh^{-1}) = ghxh^{-1}g^{-1} = (gh)x{(gh)}^{-1} = \phi_{gh}(x)$

\Problem{13} Demuestra la aserción en el ejemplo 12 que dice que los
automorfismos internos $\phi _{R_0}$, $\phi_{R_{90}}$, $\phi _{H}$ y 
$\phi _{D}$ de $D_{4}$ son distintos.

\TheSolution{}
Para mostrar que las funciones son distintas es suficiente encontrar un
elemento en el que $\phi _{1} (i) \neq \phi _{2} (i)$ para cada par de
funciones.

Para $\phi _{R_0}$ con las demas:

$$\phi _{R_0} (R_{90}) \neq \phi _{R_90} (R_{90})$$

$$\phi _{R_0} (R_{90}) \neq \phi _{R_H} (R_{90})$$

$$\phi _{R_0} (R_{90}) \neq \phi _{R_D} (R_{90})$$

Para $\phi _{R_90}$ con las demas:

$$\phi _{R_90} (R_{180}) \neq \phi _{R_H} (R_{180})$$

$$\phi _{R_90} (R_{180}) \neq \phi _{R_D} (R_{180})$$

Para $\phi _{R_H}$ con las demas:

$$\phi _{R_H} (R_{180}) \neq \phi _{R_D} (R_{180})$$

Como encontramos al menos un elemento para cada pareja en el cual no
resultan lo mismo, las 4 funciones son distintas.

\Problem{15} Si $G$ es un grupo, demuestra que $Aut(G)$ y $Inn(G)$ son grupos
[bajo la operación de composición de funciones].

\TheSolution{} Tomamos un elemento $\alpha \in Aut(G)$. Para probar que
$Aut(G)$ es un grupo, solo hace falta mostrar que $\alpha^{-1}$ preserva la
operación del $G$:
\[
\begin{split}
  \alpha^{-1}(xy)              &= \alpha^{-1}x\alpha^{-1}y \\
  \iff \alpha(\alpha^{-1}(xy)) &= \alpha(\alpha^{-1}x\alpha^{-1}y) \\
  \iff xy                      &= \alpha(\alpha^{-1}x)\alpha(\alpha^{-1}y) \\
                               &= xy
  \end{split}
\]

Por lo que $Aut(G)$ es un grupo.

\Problem{16} Demuestra que el mappeo de $U(16)$ a si mismo dado por $x
\rightarrow x^3$ es un automorfismo. ¿Qué tal $x \rightarrow x^5$ y $x
\rightarrow x^{7}$? Generalice.

Para probar que es 1-1:

Suponemos $\phi (x) = \phi (y)$, dado el automorfismo sugerido, tenemos que
$x^{3}=y^{3}$. Por lo tanto: $x=y$. Para probar que es sobre: $\phi (x) = y$.
Es decir $x^{3}=y$. Esto se vuelve más evidente al saber que hablamos del mismo
grupo y un elemento del grupo operando sobre él mismo, hace que obtengamos un
elemento del mismo grupo. Y para la conservación de la operación:

$$\phi (x\cdot y) = \phi (x)\cdot \phi (y)$$
$$\phi (x\cdot y) = x^{3}\cdot y^{3} = \phi (x)\cdot \phi (y)$$

Es decir, que $\phi$ es un automorfismo. Lo mismo se cumple para las potencias
impares, incluso cuando trabajamos con enteros negativos.

\Problem{18} ¿A qué grupo familiar es isomorfo el grupo $\left \{
\begin{bmatrix}
1 & a \\
0 & 1
\end{bmatrix} | a \in Z \right \}$? ¿Qué pasa si $Z$ es reemplazado por $R$?

\TheSolution{}
El grupo es isomorfo a $(Z, +)$. Si definimos $\phi\left(\begin{bmatrix}
1 & a \\
0 & 1
\end{bmatrix}\right) = a$, por construcción es biyectiva. Ahora, veremos que preserva operación.

\begin{align*}
\phi\left(\begin{bmatrix}
1 & a \\
0 & 1
\end{bmatrix} \cdot \begin{bmatrix}
1 & b \\
0 & 1
\end{bmatrix}\right) &= \phi\left( \begin{bmatrix}
1 & b+a \\
0 & 1
\end{bmatrix} \right) \\
&= b + a \\
&= a + b \\
&= \phi\left(\begin{bmatrix}
1 & a \\
0 & 1
\end{bmatrix}\right) + \phi\left(\begin{bmatrix}
1 & b \\
0 & 1
\end{bmatrix}\right)
\end{align*}

Si reemplazamos $Z$ por $R$, el grupo es isomorfo a $(R, +)$.

\Problem{19} Si $\phi$ y $\gamma$ son isomorfismos de un grupo ciclico $\langle
a \rangle$ a algun grupo y $\phi(a) = \gamma(a)$, demuestre que $\phi =
\gamma$.

\TheSolution{}
Sabemos que $\phi, \gamma \in Iso\langle a \rangle$ y que $\phi(a) = \gamma(a)$.

Sea $\phi(a) = \gamma(a) = t$. Tambi\'en, sea $\forall n \in \langle a \rangle
\Rightarrow n = a^n$, $n\in \mathbb Z$.

$\phi(a^n) = {(\phi(a))}^n = \phi(n) = t^n$

$\gamma(a^n) = {(\gamma(a))}^n = \gamma(n) = t^n$

$\phi(n) = \gamma(n) \ t^n$

$\therefore \phi = \gamma$

\Problem{20} Suponga que $\phi: Z_{50} \rightarrow Z_{50}$ es un automorfmismo
con $\phi(11) = 13$. Determine una formula para $\phi(x)$.

\TheSolution{}
Observamos que $\phi(11) = 11\phi(1) = 13$ y como $11^{-1}$ está en el grupo,
podemos decir que $\phi(1) = 11^{-1}13 = 41 \cdot 13 = 33$. Ahora vemos que
$\phi(x) = \phi(x\cdot 1) = x \cdot \phi(1) = 33x$.

\Problem{23} Refiriendose al teorema 6.1, demuestre que $T_g$ es en verdad una
permutación al conjunto $G$.

\TheSolution{}
Recordemos que una permutación es una función que es uno a uno y que es sobre.
Para demostrar que $T_{g}$ es uno a uno, supongamos que $T_{g}(x) = T_{g}(y)$
para $x,y$ que pertenecen al grupo G. Entonces $gx=gy$ y por lo tanto, $x=y$.
Para demostrar que es sobre, es decir, que existen $x$ e $y$ tales que
$T_{g}(x)=y$, haremos lo siguiente:

$$gx=y$$
$$x=g^{-1}y$$

Y como el inverso de $g$ también pertenece a $G$, se cumple que también es
sobre. Por lo tanto, efectivamente $T_{g}$ es una permutación.

\Problem{24} Prueba o desprueba que $U(20)$ y $U(24)$ son isomorficos.

\TheSolution{}
$U(20) = \{1,3,7,9,11,13,17,19\}$ y $U(24) = \{1,5,7,11,13,17,19,23\}$. Todo
elemento no trivial en $U(24)$ tiene orden 2 lo cual no sucede en $U(20)$. Por
lo tanto no son isomorfos.

\Problem{25} Muestre que la función $\phi (a+bi) = a -bi$ es un automorfismo
del grupo de los números complejos bajo la adición. Muestre que $\phi$ también
preserva la multiplicación de números complejos.

\TheSolution{}
Por construcción, $\phi$ es sobre. Si tenemos que $\phi(a+bi) = \phi(c+di)$,
entonces tenemos que $a-bi = c-di \rightarrow a-c = 0$ y $di - bi = 0$, por lo
que $a = c$ y $d = b$, así, es 1 a 1.

Si sumamos dos números complejos con $\phi$ tenemos
\begin{align*}
\phi(a+bi) + \phi(c+di) &= (a-bi) + (c-di) \\
&= (a+c) - (b+d)i \\
&= \phi((a+c) + (b+d)i) \\
&= \phi((a+bi)+(c+di))
\end{align*}

Si ahora multiplicamos dos números complejos tenemos
\begin{align*}
\phi(a+bi)\phi(c+di) &= (a-bi)(c-di) \\
&= ac - adi - bci - bd \\
&= (ac-bd) - (ad + bc)i \\
&= \phi((ac-bd) + (ad + bc)i) \\
&= \phi((a+bi)(c+di))
\end{align*}

\Problem{29} Sean $C$ los numeros complejos y

\[
  M = \Bigg\{
\begin{bmatrix}
  a & -b \\
  b & a
\end{bmatrix}
\Big| a, b \in R
\Bigg\}
\]

Demuestre que $C$ y $M$ son isomorficos bajo la suma y que $C*$ y $M*$, los
elemenos no cero de $M$, son isomorficos bajo la multiplicación.

\TheSolution{}
Tomamos el isomorfismo
\[
\phi(a + bi)
  =
\begin{bmatrix}
  a & -b \\
  b & a
\end{bmatrix}
\]

Revisamos la operación:

\[
\begin{split}
  \phi((a + bi) + (c + di)) &=
  \begin{bmatrix}
    a + c & -(b + d) \\
    b + d & a + c
  \end{bmatrix}
  = \\
  \begin{bmatrix}
    a & -b \\
    b & a
  \end{bmatrix}
  +
  \begin{bmatrix}
    c & -d \\
    d & c
  \end{bmatrix}
  &=
  \phi(a + bi) + \phi(c + di)
\end{split}
\]

Y para el otro caso:


\begin{align}
  \phi((a + bi)(c + di)) = \phi((ac - bd) + (ad + bc)i) = \\
  \begin{bmatrix}
    ac - bd & -(ad + bc) \\
    ac + bd & ac - bd
  \end{bmatrix}
  =
  \begin{bmatrix}
    a & -b \\
    b & a
  \end{bmatrix}
  \begin{bmatrix}
    c & -d \\
    d & c
  \end{bmatrix}
\end{align}


\Problem{33} Sea $G$ un grupo y sea $g \in G$. Si $z \in Z(G)$, muestra que
el automorfismo interno inducido por $g$ es el mismo que el automorfismo
inducido por $zg$ (esto es, que los mappeos $\phi_g$ y $\phi_{zg}$ son
iguales).

\TheSolution{} $\phi_g(x) = gxg^{-1}$ y $\phi_{zg}(x) = zgx{(zg)}^{-1} =
zgxz^{-1}g^{-1} = zz^{-1}gxg^{-1} = gxg^{-1}$ ya que $z \in Z(G)$.

\Problem{35} Supón que $g$ y $h$ inducen el mismo automorfismo interno de un
grupo $G$. Demuestra que $h^{-1}g \in Z(G)$.

\TheSolution{} Si $\phi_g = \phi_h$, entonces $gxg^{-1} = hxh^{-1}$ para toda
$x$. Entonces $x = h^{-1}gxg^{-1}h = h^{-1}gx{(h^{-1}g)}^{-1}$, lo que implica
que $h^{-1}g \in Z(G)$.

\Problem{36} Combina los resultados del ejercicio 33 y 35 en un solo teorema
``Si y solo si''.

\TheSolution{} $\phi_g = \phi_h$ si y solo si $h^{-1}g \in Z(G)$

\Problem{43} Demuestra que $Q^+$, el grupo de los numeros racionales positivos
bajo la multiplicación, es isomorfo a un subgrupo propio.

\TheSolution{} $\phi(x) = x^2$ es uno a uno, ya que si $a^2 = b^2$, $a = b$
para $a$ y $b$ en $Q^+$. La operación se preserva, ya que $\phi(ab) = (ab^2) =
a^2b^2 = \phi(a)\phi(b)$. Sin embargo, $\phi$, no es sobre, ya que no existe
numero racional cuyo cuadrado sea 2, asi que la imagen de $\phi$ forma un
subgrupo propio de $Q^+$.

\end{document}
