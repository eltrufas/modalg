\documentclass{article}

\usepackage[utf8]{inputenc}
\usepackage{amsmath}
\usepackage{mathtools}
\usepackage{enumerate}
\usepackage{listings}
\usepackage{color}

\definecolor{dkgreen}{rgb}{0,0.6,0}
\definecolor{gray}{rgb}{0.5,0.5,0.5}
\definecolor{mauve}{rgb}{0.58,0,0.82}

\lstset{frame=tb,
  language=Python,
  aboveskip=3mm,
  belowskip=3mm,
  showstringspaces=false,
  columns=flexible,
  basicstyle={\small\ttfamily},
  numbers=none,
  numberstyle=\tiny\color{gray},
  keywordstyle=\color{blue},
  commentstyle=\color{dkgreen},
  stringstyle=\color{mauve},
  breaklines=true,
  breakatwhitespace=true,
  tabsize=3
}
\newcounter{problem}
\newcounter{solution}

\newcommand\Problem[1]{%
  \stepcounter{problem}%
  \textbf{#1.}~%
  \setcounter{solution}{0}%
}

\newcommand\TheSolution{%
  \textbf{Solución:}\\%
}

\newcommand\ASolution{%
  \stepcounter{solution}%
  \textbf{Solución \thesolution:}\\%
}
\setlength{\parindent}{0in}
\setlength{\parskip}{1em}

\title{Segunda serie de ejercicios de Álgebra Moderna}
\author{Akiyuki Shinbou}
\date{Mayo 2018}

\begin{document}

\maketitle

\Problem{1} Da dos razones de porque el conjunto de numeros impares bajo la
suma no es un grupo

\TheSolution{} La suma de dos numeros impares no es otro numero impar y la
identidad $0$ no estáen el conjunto.

\Problem{2} Verifica la aserción de que la resta no es asociativa.

\TheSolution{} $(5 - 4) - 3 = -2$ pero $5 - (4 - 3) = 4$.

\Problem{3} Demuestra que $\{1, 2, 3\}$ bajo multiplicación modulo 4 no es un
grupo pero que $\{1, 2, 3, 4\}$ bajo la multiplicación modulo 5 es un grupo.

\TheSolution{} 2 no tiene un inverso bajo la multiplicación modulo 4. Bajo la
multiplicación modulo 5, $\{1, 2, 3, 4\}$ es cerrado, tiene 1 como la
identidad y todos los elementos tienen inversos: 1 y 4 son su propio inverso,
3 es inverso del 2 y 2 es el inverso del 3.

\Problem{4} Demuestra que $GL(2, R)$ es no abeliana encontrando un ejemplo de
dos matrices $A$ y $B$ en $GL(2, R)$ tal que $AB \neq BA$.

\TheSolution{} Ejemplo:
\[
\begin{bmatrix}
    1  &  0      \\
    1  &  1
\end{bmatrix}
\begin{bmatrix}
    1  &  1      \\
    0  &  1
\end{bmatrix}
\neq
\begin{bmatrix}
    1  &  1      \\
    0  &  1
\end{bmatrix}
\begin{bmatrix}
    1  &  0      \\
    1  &  1
\end{bmatrix}
\]

\Problem{5} Encuentra el elemento inverso al elemento $\begin{bmatrix}2&6\\3&5
\end{bmatrix}$ en $GL(2, Z_{11})$.


\TheSolution{} Obtenemos el determinante
\[
\det
\begin{bmatrix}
    2  &  6      \\
    3  &  5
\end{bmatrix}
= -8 \mod 11 = 3
\]

El inverso por lo tanto es

\[
\begin{bmatrix}
    5 \cdot 4  &  -6 \cdot 4      \\
    -3 \cdot 4  & 2  \cdot 4
\end{bmatrix}
=
\begin{bmatrix}
    9  &  9      \\
    10  &  8
\end{bmatrix}
\]

\Problem{6} Da un ejemplo de elementos de grupo $a$ y $b$ con la propiedad
que $a^{-1}ba \neq b$.

\TheSolution{} En $D_4$, $R_{90}VR_{270} = H$

\Problem{7} Traduce cada una de las expresiones multiplicativas en sus
contrapartes aditivas. Asume que la operación es conmutativa.

\begin{enumerate}[a.]%for small alpha-characters within brackets.
\item $a^2b^3$
\item $a^{-2}{(b^{-1}c)}^2$
\item ${(ab^2)}^{-3}c^2 = e$
\end{enumerate}

\TheSolution{}\begin{enumerate}[a.]
\item $2a + 3b$
\item $2{(c - b)} - 2a$
\item $2c - 3{a + 2b} = 0$
\end{enumerate}

\Problem{8} Demuestra que el conjunto $\{5, 15, 25, 35\}$ es un grupo bajo la
multiplicación modulo 40. ¿Cual es la identidad de este grupo? ¿Puedes ver
alguna relación entre este grupo e $U(8)$?

\TheSolution{}
La identidad del grupo es 25. El siguiente codigo de Python revisa todas las
propiedades por exhaustión.

\begin{lstlisting}
>>> G = [5, 15, 25, 35]
>>> all((x * y) % 40 in G for x in G for y in G) # Cerradura
True
>>> all((x * 25) % 40 == x for x in G) # Identidad
True
>>> all(any((x * y) % 40 == 25 for y in G) for x in G) # Existencia de la inversa
True
\end{lstlisting}

\Problem{10} Lista los miembros de $H = \{x^2 | x \in D_4$ y
$ K = \{x \in D_4 | x^2 = e \}$.

\TheSolution{}
El siguiente programa de Python enumera los dos conjuntos.

\begin{lstlisting}
tags = {(0, 1, 2, 3): 'R0',
        (1, 2, 3, 0): 'R90',
        (2, 3, 0, 1): 'R180',
        (3, 0, 1, 2): 'R270',
        (3, 2, 1, 0): 'V',
        (1, 0, 3, 2): 'H',
        (0, 3, 2, 1): 'D',
        (2, 1, 0, 3): "D'"}

D4 = tags.keys()


def compose(a, b):
    return tuple(a[i] for i in b)


H = {tags[compose(p, p)] for p in D4}
K = {tags[p] for p in D4 if tags[compose(p, p)] == 'R0'}

print("H =", H) # H = {'R180', 'R0'}
print("K =", K) # K =  {'H', 'V', 'R0', 'R180', 'D', "D'"}
\end{lstlisting}

\Problem{11} Demuestra que el conjunto de todas las matrices $2 \times 2$ con
entradas de R y determinante 1 son un grupo bajo el producto matricial.

\TheSolution{}
Por la propiedad de $\det {(AB)} = \det A \det B$, el grupo es cerrado. La
identidad es la matriz $I$.

Como $\begin{bmatrix}a&b\\c&d\end{bmatrix}^{-1} = \begin{bmatrix}d&-b\\-c&a
\end{bmatrix}$ y $\det A = ad - bc = 1$, entonces $\det A^{-1} = da -
(-b)(-c) = ad - bc = 1$ y el inverso existe en el grupo.

\Problem{12} Para cada entero n > 2, demuestra que existen al menos dos
elementos en $U(n)$ tales que $x^2 = 1$

\TheSolution{}
Primero, $1^2 = 1$ es el ejemplo obvio. Despues, por propiedades de la
aritmetica modular, ${(n - 1)}^2 = 1^2 = 1$

\Problem{13} Un profesor de algebra abstracta quiso darle a un mecanógrafo una
lista de nueve enteros que formaran un grupo bajo la multiplicación modulo 91,
asi que la lista apareció como 1, 9, 16, 22, 53, 74, 79, 81. ¿Cual entero sé
quedó afuera?

\TheSolution{}
Por la propiedad de clausura, haciendo la talacha, encontramos que $9 · 74 =
29$ y 29 no está en la lista.

\Problem{14} Sea $G$ un grupo con la siguiente propiedad: Cuando $a, b$ y $c$
pertenecen a G y $ab = ca$ entonces b = c. Prueba que G es abeliano.

\TheSolution{}
Tenemos que

\[
\begin{split}
ab & = abe \\
   & = ab(a^{-1}a) \\
   & = (aba^{-1})a
\end{split}
\]

Por hipotesis, $b = aba^{-1}$. Luego,

\[
\begin{split}
ba & = (aba^{-1})a \\
   & = ab(a^{-1}a) \\
   & = ab
\end{split}
\]

Por lo tanto, G es abeliano.

\Problem{15} Sean $a$ y $b$ elementos de un grupo abeliano y n cualquier entero.
Demuestre que ${(ab)}^n = a^{n}b^n$. ¿Esto tambien es verdad para grupos no
abelianos?

Cuando $n \geq 0$ utilizamos inducción. Para el caso base, ${(ab)}^0 = e$ y
$a^{0}b^{0} = (e)(e) = e$. En el paso inductivo, ${(ab)}^{n + 1} = {(ab)}^{n}ab =
a^{n}b^{n}ab = a^{n + 1}b^{n + 1}$. Para $n < 0$, decimos que $e = {(ab)}^0 =
{(ab)}^{n}{(ab)}^{-n} = a^{n}b^{n}{(ab)}^{-n}$. De aqui procede que ${(ab)}^n =
a^{n}b^{n}$.

\Problem{17} Demuestre que un grupo $G$ es abeliano si y solo si ${(ab)}^{-1} =
a^{-1}b^{-1}$ para toda $a$ y $b$ en $G$.

\TheSolution{} Si G es abeliano entonces ${(ab)}^{-1} = a^{-1}b^{-1}$
por la propiedad demostrada en el problema 15.

Si ${(ab)}^{-1} = a^{-1}b^{-1}$, entonces $e = {(ab)}{(ab)}^{-1} = aba^{-1}b^{-1}$.
De aqui sabemos que $b = aba^{-1}$. Multiplicando por a por la derecha obtenemos que
$ba = ab$ y por tanto G es abeliano.

\Problem{18} Demuestre que en un grupo, ${(a^{-1})}^{-1} = a$ para toda $a$.

\TheSolution{} Sabemos que $a^{-1}{(a^{-1})}^{-1} = e$. Multiplicando por $a$ por la
izquierda obtenemos que ${(a^{-1})}^{-1} = a$.

\Problem{19} Para cualesquiera elementos $a$ y $b$ de un grupo y cualquier entero
$n$, demuestra que ${(a^{-1}ba)}^{n} = a^{-1}b^{n}a$.

\TheSolution{} Se cumple trivialmente cuando $n = 0$. Cuando $n > 0$, ${(a^{-1})}^{n}
= {(a^{-1}ba)}^{n} = {(a^{-1}ba)}{(a^{-1}ba)}\ldots{(a^{-1}ba)}$ $n$ veces. Se
pueden asociar los pares de $(aa^{-1})$ para eliminarlos y llegar a que
${(a^{-1}ba)}^{n} = a^{-1}b^{n}a$. Cuando $n < 0$, observamos que $e =
{(a^{-1}ba)}^{n}{(a^{-1}ba)}^{-n} = a^{-1}b^{n}a{(a^{-1}ba)}^{-n}$. Multiplicando
ambos lados por ${(a^{-1}ba)}^{n}$ por la derecha, llegamos a que ${(a^{-1}ba)}^{n} =
a^{-1}b^{n}a$.

\Problem{20} Si $a_1,a_2,\ldots,a_n$ pertenecen a un grupo, ¿Cual es la inversa de
$a_1,a_2,\ldots,a_n$?

\TheSolution{} $a_n^{-1},a_{n - 1}^{-1},\ldots,a_2^{-1},a_1^{-1}$.

\Problem{22} Da un ejemplo de un grupo con 105 elementos. Dá dos ejemplos de
grupos con 44 elementos.

\TheSolution{} $Z_{105}$ tiene 105 elementos. $Z_{44}$ y $D_{22}$ tienen 44 elementos.

\Problem{26} Demuestra que si ${(ab)}^2 = a^{2}b^{2}$ entonces $ab = ba$.

\TheSolution{} Reescribimos la premisa como $aabb = abab$ multiplicamos por $a^{-1}$
por la izquierda y $b^{-1}$ por la derecha y obtenemos que $ab = ba$.

\Problem{27} Sean $a$, $b$ y $c$ elementos de un grupo. Resuelve la ecuación $axb = c$
para $x$. Resulelva $a^{-1}xa = c$ para $x$.

\TheSolution{}
\[
\begin{split}
  axb            & = c \\
  a^{-1}axb = xb & = a^{-1}c \\
  xbb^{-1} = x   & = a^{-1}cb^{-1}
\end{split}
\]

\[
\begin{split}
  a^{-1}xa       & = c \\
  aa^{-1}xa = xa & = ac \\
  xaa^{-1} = x   & = aca^{-1}
\end{split}
\]

\Problem{29}Sea $G$ un grupo finito. Muestra que el numero de elementos $x$ de
$G$ tales que $x^3 = e$ es impar. Muestra que el numero de elementos $x$ de
$G$ tales que $x^2 \neq e$ es par.

\TheSolution{} Primero vemos que si tenemos una solución $a$ tal que
$a^3 = e$ y $a \neq e$, tambien tenemos que ${(a^{-1})}^3 = e$. Ademas, si
tenemos una solución $b^3 = e$ tal que $b \neq e$, $b \neq a$ y
$b \neq a^{-1}$, sabemos que existe otra solución $b^{-1} = e$ tal que
$b^{-1} \neq e$, $b^{-1} \neq a$, $b^{-1} \neq b$, y $b^{-1} \neq a^{-1}$.
Por lo tanto, concluimos que las soluciones a $x^3 = e$ que no son la
identidad vienen en pares. Agregando la identidad, tenemos un numero impar
de soluciones totales.

Para la segunda pregunta, sabemos que si $x^2 \neq e$, entonces $x \neq
x^{-1}$. Luego podemos aplicar la misma logica que en el caso anterior
para probar que las soluciones a la ecuación vienen en pares, con la
diferencia que ahora la identidad no la satisface, dandonos un numero
par de soluciones.

\Problem{36} Demuestre que el conjunto de matrices $3 \times 3$ con de la
forma

\[
\begin{bmatrix}
    1  &  a & b    \\
    0  &  1 & c \\
    0  &  0 & 1
\end{bmatrix}
\]

es un grupo. (La multiplicación esta definida por

\[
\begin{bmatrix}
    1  &  a & b    \\
    0  &  1 & c \\
    0  &  0 & 1
\end{bmatrix}
\begin{bmatrix}
    1  &  a' & b'    \\
    0  &  1 & c' \\
    0  &  0 & 1
\end{bmatrix}
=
\begin{bmatrix}
    1  &  a + a' & b' + ac' + b    \\
    0  &  1 & c' + c \\
    0  &  0 & 1
\end{bmatrix}
\]

Este grupo, a veces llamado el \textit{Grupo de Heisenberg} en honor al fisico
ganador del Nobel Werner Heisenberg, y esta intimamente relacionado al
principio de incertidumbre de Heisenberg en la fisica cuantica.)

\TheSolution{} Por definición se cumple la cerradura. La identidad es la
matríz $I$. La inversa de la matriz con entradas $a$, $b$ y $c$ se obtiene
encontrando los $a'$, $b'$ y $c'$ tales que

\[
\begin{split}
  a + a'       & = 0 \\
  b' + ac' + b & = 0 \\
  c' + c       & = 0
\end{split}
\]



\end{document}
