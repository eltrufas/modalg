\documentclass{article}

\usepackage[utf8]{inputenc}
\usepackage{amsmath}
\usepackage{mathtools}
\usepackage{enumerate}
\usepackage{listings}
\usepackage{color}

\definecolor{dkgreen}{rgb}{0,0.6,0}
\definecolor{gray}{rgb}{0.5,0.5,0.5}
\definecolor{mauve}{rgb}{0.58,0,0.82}

\lstset{frame=tb,
  language=Python,
  aboveskip=3mm,
  belowskip=3mm,
  showstringspaces=false,
  columns=flexible,
  basicstyle={\small\ttfamily},
  numbers=none,
  numberstyle=\tiny\color{gray},
  keywordstyle=\color{blue},
  commentstyle=\color{dkgreen},
  stringstyle=\color{mauve},
  breaklines=true,
  breakatwhitespace=true,
  tabsize=3
}
\newcounter{problem}
\newcounter{solution}

\newcommand\Problem[1]{%
  \stepcounter{problem}%
  \textbf{Problema #1.}~%
  \setcounter{solution}{0}%
}

\newcommand\TheSolution{%
  \textbf{Solución:}\\%
}

\newcommand\ASolution{%
  \stepcounter{solution}%
  \textbf{Solución \thesolution:}\\%
}
\setlength{\parindent}{0in}
\setlength{\parskip}{1em}


\title{Tercera serie de ejercicios de Álgebra Moderna}
\author{Akiyuki Shinbou}
\date{Mayo 2018}

\begin{document}

\maketitle

\Problem{1} Encuentra todos los generadores de $Z_6$, $Z_8$, $Z_{20}$

\TheSolution{}
Para $Z_6$: $\{1, 5\}$

Para $Z_8$: $\{1, 3, 5, 7\}$

Para $Z_{20}$: $\{1, 3, 5, 7, 11, 13, 17, 19\}$

\Problem{2} Suponga que $\langle a \rangle$, $\langle b \rangle$ y $\langle c
\rangle$ son grupos ciclicos de orden 6, 8 y 20 respectivamente. Encuentra
todos los generadores de $\langle a \rangle$, $\langle b \rangle$ y $\langle c
\rangle$.

Para $\langle a \rangle$, los generadores son $\{a^{1}, a^{5}\}$

Para $\langle b \rangle$, los generadores son $\{b^{1}, b^{3}, b^{5}, b^{7}\}$

Para $\langle c \rangle$, los generadores son $\{c^{1}, c^{3}, c^{5}, c^{7},
c^{11}, c^{13}, c^{17}, c^{19}\}$

\Problem{3} Enlista los elementos de los subgrupos $\langle20\rangle$ y
$\langle 10 \rangle$ en $Z_{30}$. Sea $a$ el elemento del grupo de orden 30.
Enlista los elementos de los subgrupos $a^{20}$ y $a^{10}$.

\[
\begin{align*}
  \langle 20 \rangle &= \{0, 10, 20\} \\
  \langle 10 \rangle &= \{0, 10, 20\} \\
\end{align*}
\]

Para $a$

\[
\begin{align*}
  |a| = |\langle a \rangle| &= 30 \\
  mcd(30,20)  &= mcd(30,10) \\
              &= 1 0\\
  |\langle a^{10}\rangle|  &= |\langle a^{20}\rangle| \\
                           &= 30 / 10 = 3 \\
  \langle a^{10}\rangle
  = \langle a^{20} \rangle &= \langle a^{mcd(30, 10)} \rangle \\
                           &= \langle a^{mcd(30, 20)}\rangle \\
                           &= \langle a^{10}\rangle \\
	\langle a^{10} \rangle   &= \{e, a^{10}, a^{20}\}
\end{align*}
\]

\Problem{4} Enlista los elementos de los subgrupos $\langle 3 \rangle$ y
$\langle 15 \rangle$ en $Z_{18}$. Sea $a$ un elemento de grupo de orden
18. Enlista los elementos de los subgrupos $\langle a^3 \rangle$ y
$\langle a^{15} \rangle$.

\TheSolution{}
\[
\begin{align}
  \langle 3 \rangle  &= \{3,6,9,12,15,0\} \\
  \langle 15 \rangle &= \{15,12,9,6,3,0\} \\
  \langle a^3 \rangle &= \{a3,a6,a9,a12,a15,a0\} \\
  \langle a^{15} \rangle &= \{a15, a12, a9, a6, a3, a0\} 
\end{align}
\]

\Problem{5} Enlista los elementos de los subgrupos $\langle3\rangle$ y
$\langle 7 \rangle$ en $U(20)$

\TheSolution{}

\[U(20) = \{1,3,7,9,11,13,17,19\}\]

\[
\begin{flalign*}
  &3^1 = 3 \\
  &3^2 = 9 \\
  &3^3 = 27 \mod 20 = 7 \\
  &3^4 = 7 \cdot 3 \mod 20 = 1 \\
  &3^5 = 3\cdot1 \\
  &\langle3\rangle = \{3,7,9,1\} \\ \\
  &7^1 = 7 \\
  &7^2 = 49 \mod 20 = 9 \\
  &7^3 = 9\cdot7 \mod 20 = 3 \\
  &7^4 = 3 \cdot 7 \mod 20 = 1 \\
  &\langle7\rangle = \{1,3,7,9\}
\end{flalign*}
\]

\Problem{6} ¿Qué tienen en comun los ejercicios 3, 4 y 5? Intenta realizar una
generalización que incluya los 3 casos.

\TheSolution{}
En cualquier grupo, $\langle a \rangle = \langle a^{-1} \rangle$.

\Problem{7} Encuentra un ejemplo de un grupo no ciclico cuyos subgrupos
propios son ciclicos.

\TheSolution{}
$U(8)$.

\Problem{8} Sea $a$ un elemento de un grupo y sea $|a| = 15$. Computa el orden
de los siguientes elementos de $G$.
\begin{enumerate}[a.]
  \item $a^3, a^6, a^9, a^{12}$
  \item $a^5, a^{10}$
  \item $a^2, a^4, a^8, a^{14}$
\end{enumerate}

\TheSolution{}
\begin{enumerate}[a.]
  \item $5$
  \item $3$
  \item $15$
\end{enumerate}

\Problem{9} ¿Cuantos subgrupos tiene $Z_{20}$? Enlista los generadores de cada
uno de estos subgrupos. Suponga que $G = \langle a \rangle$ y $|a| = 20$.
¿Cuantos de estos subrupos tiene $G$? Enlista los generadores de estos subgrupos.

\TheSolution{}
Usando la propiedad de los grupos, $\langle a\rangle$ es sugrupo de $G$,
entonces contamos con $20$ subgrupos de $G$, donde el generador es obvio. De
igual manera, ya que el orden de $a$ es 20, eso significa que para obtener la
identidad se tuvo que operar y resultar en otros 19 elementos, donde cada uno
de estos es generador de otro subgrupo de $G$. Es decir, que también existen
$20$ subgrupos y cada potencia de $a$ es un generador.

\Problem{10} En $Z_{24}$, enlista todos los generadores para el subgrupo de
orden 8. Sea $G = \langle a \rangle$ y sea $|a| = 24$. Enlista todos los
generadores del subgrupo de orden 8.

\TheSolution{}

\[
\begin{align*}
    Z_{24} &= \{0,1,\ldots,23\} \\
		\langle 24/8 \rangle = \langle 3 \rangle &= {0, 3, 6, 9, 12, 15, 18, 21}
    \\
		|3| &= |\langle 3 \rangle| = 8
\end{align*}
\]
Generadores para $Z_{24}$ considerando el producto mod 24: $\{3,9,15,21\}$
obtenidos a partir de la fórmula: $|a|=\frac{n}{mdc(n,k)}$ recorriendo los
números desde 0 a 24 donde 24 es el orden de $Z_{24}$

Ahora

\[
\langle a \rangle  = \{e, a, a^{2}, \ldots, a^{23}\}
\]

Subgrupo de orden 8

\[
\begin{align*}
  \langle a^{\frac{24}{8}}\rangle &= \langle a^{3} \rangle \\
  \langle a^{3} \rangle &= \{e, a^{3}, a^{6}, a^{9}, a^{12}, a^{15}, a^{18}, a^{21}\}
\end{align*}
\]

Generadores

\[
\langle a^{3} \rangle = \{a^{3}, a^{9}, a^{15}, a^{21}\}
\]

Obtenidos de la misma forma que el inciso anterior

\Problem{11} Sea $G$ un grupo y $a \in G$. Demuestre que $\langle a^{-1}
\rangle = \langle a \rangle$.

\TheSolution{}
$a^{-1} \in \langle a \rangle$ por definicion, asi que $\langle a^{-1}
\subseteq \langle a \rangle$. Por otro lado, $a = {(a^{-1})}^{-1} \in \langle
a^{-1} \rangle$, por lo que $\langle a \rangle \subseteq \langle a^{-1}
\rangle$. Con estas dos relaciones ya tenemos que $\langle a \rangle =
\langle a^{-1} \rangle$.

\Problem{12} En $Z$ encuentra todos los generadores del subgrupo $\langle 3
\rangle$. Si $a$ tiene orden infinito, encuentra todos los generadores del
subgrupo $\langle a^3 \rangle$.

\TheSolution{}
En $Z$, los generadores de $\langle3\rangle$ son $\{3,-3\}$

Si $a$ tiene orden infinito entonces $\langle a^3\rangle$ tiene dos generadores
$\{a^3, a^{-3}\}$.

\Problem{14} Supón que un grupo ciclico $G$ tiene exactamete tres subgrupos:
el mismo G, $\{e\}$ y un subgrupo de orden 7. ¿Cual es $|G|$? ¿Qué puedes
decir si 7 es reemplazado por $p$ donde $p$ es primo?

\TheSolution{}
Para poder tener exactamente 3 subgrupos, $|G|$ solo puede ser dividido por
tres números: 1, 7 y $|G|$. Tambien sabemos que $|G| / 7$ tambien es divisor
de G, y que este numero debe ser 7 para que solo haya 3 subgrupos, por lo que
$|G| = 7\cdot7 = 49$.

\Problem{15} Sea $G$ un grupo Abeliano y sea $H = \{g \in G | |g| $ divide a $12
\}$. Demuestra que $H$ es un subgrupo de $G$. ¿Hay algo especial sobre el 12
aqui? ¿Tu demostración sería valida si 12 fuera reemplazada por algun otro
entero positivo? Enuncia el resultado general.

\TheSolution{}
Si $|g|$ es dividido por $12$, entonces $g^{12} = e$. Sean $a$ y $b$ en $H$.
Vemos que ${(ab^-{1})}^{12} = a^{12}{(b^{12})}^{-1} = ee^{-1} = e$, por tanto,
$H$ es subgrupo de $G$.

\Problem{16} Encuentra una colección de subgrupos distintos $\langle a_{1}
\rangle ,\langle a_{2}\rangle , \ldots ,\langle a_{n} \rangle$ of $\Z_{240}$
con la propiedad que $\langle a_{1}\rangle \subset \langle a_{2}\rangle
\subset \cdots \subset \langle a_{n} \rangle$ con la $n$ mas grande posible.

\TheSolution{}
\[
\langle a_{2} \rangle, \langle a_{4} \rangle, \langle a_{8} \rangle, \langle a
_{16} \rangle \cdots \langle a_{240}\rangle
\]

\Problem{17} Completa el siguiente enunciado: $|a| = |a^2|$ si y solo si $|a|
\ldots$

\TheSolution{}
$\ldots$ es impar o infinito.

Dado que $k = 2$

\[
  |a^{2}| = \frac{n}{mdc(n,2)} = n
\]
	
Para ello se necesita que $n$ sea impar o infinito.

\Problem{18} Si un grupo ciclico tiene un elemento de orden infinito, ¿cuantos
elementos de orden finito tiene?

\TheSolution{} Uno, la identidad.

\Problem{19} Enlista los subgrupos ciclicos de $U(30)$.

\TheSolution{}
\[
  U(30) = \{1, 7, 11, 13, 17, 19, 23, 29\}
\]

\[
\begin{flalign*}
  &\langle 1 \rangle = \{1,7,19, 13\} \\
  &\langle 7 \rangle = \{1, 11\} \\
  & \langle 13 \rangle = \{1, 13, 19, 7\} \\
  & \langle 17 \rangle = \{1, 17, 19, 23\}  \\
  & \langle 19 \rangle = \{1, 19\} \\
  & \langle 23 \rangle = \{1, 23, 19, 17\} \\
  & \langle 29 \rangle = \{1, 29\} \\
\end{flalign*}
\]

\Problem{22} Demuestra que un grupo de orden 3 debe ser ciclico.

\TheSolution{}
Sea $G = \{e, a, b\}$. $ab$ debe estar en $G$. $ab$ no puede ser $a$ o $b$
porque implicaria que $a = e$ o $b = e$. Por tanto, $ab = e$ y $b = a^{-1}$,
dejando el grupo como $\{e, a, a^{-1}\}$, un grupo ciclico.

\Problem{24} Para cada elemento $a$ en cualquier grupo $G$, demuestra que
$\langle a \rangle$ es un subgrupo de $C(a)$.

\TheSolution{}
Dado que $\langle a \rangle$ se compone de los elementos de la forma $a^{n}$
entonces $a \cdot a^{n}$ = $a^{1 + n}$ = $a^{n + 1}$ = $a^{n} \cdot a$ por lo
que $a^{n}$ està en el centrizador de $a$ y como fue elegido al azar $<a>
\subset C(a)$ y como ambos estàn en G entonces es un subgrupo de $C(a)$.

\Problem{26} Encuentra todos los generadores de $Z$. Sea $a$ un elemento de
grupo con orden infinito. Encuentra todos los generadores de $\langle a
\rangle$.

\TheSolution{}
Los generadores de $Z$ son $\{-1, 1\}$.

Si $a \in G$ y $|a| = \infty$, entonces los generadores de $a$ son $\{a^{-1},
a^1\}$.

\Problem{29} Enlista los elementos de orden 8 en $Z_{8000000}$. ¿Como sabes
que tu lista esta completa? Sea $a$ un elemento del grupo tal que $|a| =
8000000$. Enlista todos los elementos de orden 8 en $\langle a \rangle$.

\TheSolution{} Por el teorema 4.3, sabemos que $\langle 1000000\rangle$ es el
unico subgrupo de orden 8, asi que sus elementos son los unicos de orden 8:
1000000, 2000000, 3000000, 4000000, 5000000, 6000000, 7000000.

\Problem{30} Supón que $a$ y $b$ pertenecen a un grupo, $a$ tiene orden impar
y $aba^-1 = b^{-1}$. Demuestra que $b^2 = e$.

\[
\begin{split}
  aba^{-1}  &= b^{-1}  \\
  ab        &= b^{-1}a  \\
  baba^{-1} &= e  \\
  a         &= baba \\ 
  a^{-1}    &= b^{-2}a^{-2} \\
  a         &= b^{-2} \\
  ab        &= b^{-1} \\
  a         &= e \\
  b         &=e=b^{2} \\ 
\end{split}
\]

\Problem{31} Sea $G$ un grupo finito. Muestra que existe un numero fijo de
enteros positivos $n$ tales que $a^n = e$ para todo $a$ en $G$.

\TheSolution{}
$G$ es $\{e, a_{1}, a_{2}, \ldots, a_{k}\}$, entonces $|a_{i}|=n_{i}$. Para
encontrar la $n$ general calculamos $n = lcm(n_{1}, n_{2}, \ldots, n_{k})$.
Ahora $a^{n} = a^{kn} = e$, $\forall a \in G$.

\Problem{34} Determine la cuadricula de subgrupos para $Z_8$

\TheSolution{}
Orden 8:
$\langle 1 \rangle = {0, 1, 2, 3, 4, 5, 6, 7}$

Orden 4:
$\langle 2 \rangle = {0, 2, 4, 6}$

Orden 2:
$\langle 4 \rangle = {0, 4}$

Orden 1:
$\langle 8 \rangle = {0}$

\Problem{36} Demuestra que un grupo finito es la union de subgrupos propios si
y solo si el grupo no es ciclico.

\TheSolution{}
Supongamos que el grupo es la union de grupos propios. Si $G = \langle
a\rangle$, entonces $a$ pertenece a algun subgrupo propio $H_i$ de $G$ y
por tanto $\langle a \rangle = G$ debe ser un subgrupo de $H_i$, lo cual es
una contradicción.

Por otro lado, asumimos que $G$ no es ciclico. Tomamos un elemento $a$
cualquiera de $G$. Como $G$ no es ciclico, $\langle a \rangle$ es un subgrupo
propio de $G$. Podemos repetir este proceso para todos los elementos $a$ y
al unir estos subgrupos propios obtenemos todo $G$.

\Problem{37} Demuestra que el grupo de numeros enteros racionales bajo la
multiplicación no es ciclica.

\TheSolution{}
Supongamos que existe un generador para el grupo, y ese es de la forma
$\frac{a}{b}$. Para que este sea generador, debe generar todo el grupo a
partir de potencias, lo cual sería imposible en el grupo de los racionales
positivos, ya que una fracción de la forma $\frac{a}{b}^{n}$ va a tender hacia
infinito o hacia $0$, pero todo con una tendencia muy marcada. Es decir,
siempre sus potencias van a ser múltiplos de este o de su inverso. 

\Problem{38} Considera el conjunto $\{4, 8, 12, 16\}$. Demuestra que este
conjunto es un grupo bajo la multiplicación modulo 20 construyendo su tabla
de Cayley. ¿Cual es el elemento identidad? ¿El grupo es ciclico? Si lo es,
encuentra los generadores.

\TheSolution{}
\begin{center}
    	\begin{tabular}{c|cccc}
             & 4 & 8 & 12 & 16 \\ 
           \hline 
           4 & 16 & 12 & 8 & 4 \\ 
           8 & 12 & 4 & 16 & 8 \\ 
           12 & 8 & 16 & 4 & 12 \\ 
           16 & 4 & 8 & 12 & 16 \\ 
		 \end{tabular}
    \end{center}
La identidad es 16. El grupo es ciclico pues $\langle 8,12 \rangle = \{4,8,12,
16\}$

\Problem{40} Sean $m$ y $n$ elementos del grupo $Z$. Encuentra los generadores
del grupo $\langle m \rangle \cap \langle n \rangle$.

\TheSolution{}
$m,n \in Z$.\\
	Suponiendo que $\langle m \rangle \cap \langle n \rangle = \langle lcm(m,n) \rangle$

	$\langle m \rangle \cap \langle n \rangle \leq Z \Rightarrow \langle m \rangle \cap \langle n \rangle = \langle k \rangle$ para alguna $k \in Z$\\
	$\Rightarrow k = m \cdot k_1$, y $k = n \cdot k_2$ para alguna $k_1,k_2 \in Z$\\
	$\Rightarrow lcm(m,n)$ divide a $k$\\
	Sea $lcm(m,n) \in \langle m \rangle \cap \langle n \rangle$, y desde que $m,n$ divide a $lcm(m,n)$\\
	$\rightarrow lcm(m,n) \in \langle k \rangle$\\
	$\rightarrow lcm(m,n) = k \cdot q$ para alguna $q \in Z$\\
	$\rightarrow k$ divide a $ lcm(m,n)$\\
	Entonces $k = lcm(m,n)$ y esto es que $\langle m \rangle \cap \langle n \rangle = \langle lcm(m,n) \rangle$.\\

\Problem{41} Suponga que $a$ y $b$ son elementos de grupo que conmutan y
tienen ordenes $m$ y $n$. Si $\langle a \rangle \cap \langle b \rangle = {e}$,
demuestra que el grupo contiene un elemento cuyo orden es al menos el minimo
comun multiplo de $m$ y $n$. Demuestra que esto no necesita ser verdad si $a$
y $b$ no conmutan.

\TheSolution{}
Suppose that a and b are group elements that commute and have
orders m and n. If $\langle a \rangle \cap \langle b \rangle = {e}$ ,
prove that the group contains an element whose order is the least common multiple of m and n.
Show that this need not be true if a and b do not commute.

De la hipotesis, el orden de $ab$ es lcm(m, n).

Si consideramos $|ab| = r$ entonces:
\[
\begin{align}
  {(ab)}^{r} &= a^{r} b^{r} = e \\
  a^{r}      &= b^{-r} \in \langle b \rangle \\
  a^{r} \in \langle a \rangle \cap \langle b \rangle &= {e} \\
  a^{r} &= e \\
  b^{r} &= e
\end{align}
\]

Por estos resultados, $m | r$ y $n | r$ y por lo tanto $lcm(m, n) | r$,
y en particular, $r \geq lcm(m, n)$

Ahora consideremos $q = lcm(m, n)$. Por lo tanto q se puede descomponer en 
$m q_{1}$ y $n q_{2}$ para $q_{1}, q_{2}$ positivos.

Revisitando una de las formulas anteriores tenemos:

\[
\begin{align}
  {(ab)}^{q} &= a^{q} b^{q} \\
  a^{q} b^{q} &= a^{m q_{1}} b^{n q_{2}} \\
  a^{m q_{1}} b^{n q_{2}} &= {(a^{m})}^{q_{1}} {(b^{n})}^{q_{2}} \\
  {(a^{m})}^{q_{1}} {(b^{n})}^{q_{2}} &= e^{q_{1}} e^{q_{2}} = e
\end{align}
\]

Asi que $r = |ab| \leq q = lcm(m, n)$. Por lo tanto $r = lcm(m, n)$.

Ahora veamos el caso en el que a y b no conmutan.
En este caso, no necesariamente existe r que cumpla lo que se quiere
y se puede ver con un ejemplo.
Veamos el caso de $S_{3}$.
Si $a = (12)$ y $b = (123)$ entonces $|a| = 2$ y $|b| = 3$
pero porque $S_{3}$ no es abeliano, no es un grupo ciclico y no existe un elemento
de grado $lcm(2, 3) = 6$.

\Problem{43} Sea $p$ un primo. Si un grupo tiene mas de $p - 1$ elementos de
orden $p$, ¿Por qué el grupo no puede ser ciclico?

\TheSolution{}
Supongamos un grupo ciclico finito. Si tenemos un subgrupo de orden
$p$, este tendra $\phi(p) = p - 1$ elementos de orden $p$ por el teorema 4.4.
Si existiera otro elemento de orden $p$, tendria que existir otro subgrupo de
orden $p$, lo cual contradice la propiedad de los subgrupos finitos
establecida en el teorema 4.3.

\Problem{45} Enlista todos los elementos de $Z_{40}$ de orden 10. Sea $|x| = 
40$. Enlista todos los elementos de $x$ que tienen orden 10.

\TheSolution{}
\begin{enumerate}
  \item Los de orden 10 en $Z_{40}$ = $\{4,12,28,36\}$ \\
    Obtenidos a partir de la f\'ormula $|a| = \frac{n}{mcd(n, k)}$ se
    seleccionan aquellos n\'umeros en $Z_{40}$ tales que $mcd(40, x) = 4$
    desde 0 a 39.
  \item Los de orden 10 en $<x>$ = $\{1,3,7,9\}$ donde $|x| = 10$ \\
    Similar al anterior pero ahora se seleccionan sólo aquellos $mcd(10, x) =
    1$ desde 0 a 9
\end{enumerate}


\Problem{47} Determina los ordenes de los elementos de $D_{33}$ y cuantos hay
de cada uno.

\TheSolution{}
$D_{33}$ tiene 33 ejes de simetria que forman 33 permutaciones de orden 2.
Tambien hay 33 rotationes, los cuales forman un grupo ciclico. Sabemos que
por cada divisor $d$ de 33 tenemos $\phi(d)$ rotaciones de orden $d$ en el
grupo. Asi obtenemos $\phi(1) = 1$ elemento de orden 1, $\phi(3) = 2$
elementos de orden 3, $\phi(11) = 10$ elementos de orden 11 y $\phi(33) =
20$ elementos de orden 33.


\Problem{50} Si G es un grupo Abeliano y contiene subgrupos ciclicos de
ordenes 4 y 6, ¿Qué otros tamaños de subgrupos ciclicos estan contenidos en
$G$? Generalice.

\TheSolution{} Tomamos dos grupos ciclicos $|\langle a \rangle| = 4$ y
$|\langle b\rangle| = 6$. Sabemos que ${(a^4)}^{3}{(b^6)}^2 = ee = e$, asi que
sabemos que 12 divide a $|ab|$. Tambien sabemos que ${(ab)}^{4} = a^4 = e$,
por lo tanto, $|ab| \neq 1, 2$ o $4$. Conversamente, vemos que ${(ab)}^{6} =
a^{6} \neq e$, asi que $|ab| \neq 1, 2, 3, 6$. Por tanto, debe en existir
subgrupos de orden $1, 2, 3, 4, 6$ y $12$.

\Problem{54} Sean $a$ y $b$ en un grupo. Si $|a|$ y $|b|$ son primos
relativos, demuestra que $\langle a \rangle \cap \langle b \rangle = \{e\}$

\TheSolution{}
Sea $a,b \in G$

Suponiendo que $ \langle a \rangle \cap \langle b \rangle  \leq \langle a
\rangle$ y  $ \langle a \rangle \cap \langle b \rangle  \leq \langle b
\rangle$.\\\\
Si $x \in \langle a \rangle \cap \langle b \rangle$\\ entonces $|x| |a|$ y
$|x||b|$,\\
$|x| = \gcd(|a|,|b|)$.\\
$|x| = 1$ porque $gcd(|a|,|b|) = 1$\\
$\{x\} = \{e\}$.\\
$\langle a \rangle \cap \langle b \rangle = \{e\}$

\Problem{57} Supón que un grupo $G$ tiene al menos nueve elmentos $x$ tales
que $x^8 = e$. ¿Puedes concluir que $G$ no es ciclico? ¿Y si $G$ tiene al
menos cinco elementos $x$ tales que $x^4 = e$? Generalice.

\TheSolution{} Sabemos que $|x|$ divide a $8$. Si $G$ fuese ciclico, entonces
habria exactamente $\phi(8) + \phi(4) + \phi(2) + \phi(1) = 4 + 2 + 1 + 1 = 8$
elementos con ordenes que dividan a $8$ en $G$, asi que $G$ no es ciclico.
Igual en el otro caso, habria $\phi(4) + \phi(2) + \phi(1) = 2 + 1 + 1 = 4$
elementos con ordenes que dividan a 4, por lo que $G$ tampoco seria ciclico.

\Problem{61} Supon que $a$ es un elemento de grupo tal que $|a^{28}| = 10$ y
$|a^{22}| = 20$. Determine $|a|$.

\TheSolution{}
$|a^{28}| = 10 \Rightarrow a^{280} = e$\\
$|a^{22}| = 20 \Rightarrow a^{440} = e$\\
\\
$|a|$ divide a 280 y $|a|$ divide a 440.\\
$|a|$ divide a $\gcd(280,440)$.\\
$|a|$ divide a 40.\\
$|a| = 40$.\\

\Problem{64} Demuestre que $H = \Bigg\{\begin{bmatrix}
  1 & n \\
  0 & 1
\end{bmatrix} | n \in Z\Bigg\}$ es un subgrupo ciclico de $GL(2, R)$

\TheSolution{} Observamos que $\begin{bmatrix}
  1 & n \\
  0 & 1
\end{bmatrix} = \begin{bmatrix}
  1 & 1 \\
  0 & 1
\end{bmatrix}^{n}$, que pertenece a $GL(2, R)$, por lo que H es un subgrupo de
este.

\Problem{68} Sean $r_1$ y $r_2$ numeros racionales. Demuestre que el grupo
$G = \{n_1r_1 + n_2r_2 | n_1, n_2 \in Z \}$ bajo la suma es ciclico.
Generalice al caso donde tienes $r_1$, $r_2$, $\ldots$, $r_k$ rationals.

\Problem{71} Demuestra que para cualquier primo $p$ y entero positivo $n$,
$\phi(p^n) = p^n - p^{n - 1}$.


\end{document}
