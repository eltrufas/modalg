\documentclass{article}

\usepackage[utf8]{inputenc}
\usepackage{amsmath}
\usepackage{mathtools}
\usepackage{enumerate}
\usepackage{listings}
\usepackage{color}

\definecolor{dkgreen}{rgb}{0,0.6,0}
\definecolor{gray}{rgb}{0.5,0.5,0.5}
\definecolor{mauve}{rgb}{0.58,0,0.82}

\lstset{frame=tb,
  language=Python,
  aboveskip=3mm,
  belowskip=3mm,
  showstringspaces=false,
  columns=flexible,
  basicstyle={\small\ttfamily},
  numbers=none,
  numberstyle=\tiny\color{gray},
  keywordstyle=\color{blue},
  commentstyle=\color{dkgreen},
  stringstyle=\color{mauve},
  breaklines=true,
  breakatwhitespace=true,
  tabsize=3
}
\newcounter{problem}
\newcounter{solution}

\newcommand\Problem[1]{%
  \stepcounter{problem}%
  \textbf{Problema #1.}~%
  \setcounter{solution}{0}%
}

\newcommand\TheSolution{%
  \textbf{Solución:}\\%
}

\newcommand\ASolution{%
  \stepcounter{solution}%
  \textbf{Solución \thesolution:}\\%
}
\setlength{\parindent}{0in}
\setlength{\parskip}{1em}


\title{Tercera serie de ejercicios de Álgebra Moderna}
\author{Akiyuki Shinbou}
\date{Mayo 2018}

\begin{document}

\maketitle

\Problem{1} Encuentra todos los generadores de $Z_6$, $Z_8$, $Z_{20}$

\TheSolution{}
Para $Z_6$: $\{1, 5\}$

Para $Z_8$: $\{1, 3, 5, 7\}$

Para $Z_{20}$: $\{1, 3, 5, 7, 11, 13, 17, 19\}$

\Problem{2} Suponga que $\langle a \rangle$, $\langle b \rangle$ y $\langle c
\rangle$ son grupos ciclicos de orden 6, 8 y 20 respectivamente. Encuentra
todos los generadores de $\langle a \rangle$, $\langle b \rangle$ y $\langle c
\rangle$.

Para $\langle a \rangle$, los generadores son $\{a^{1}, a^{5}\}$

Para $\langle b \rangle$, los generadores son $\{b^{1}, b^{3}, b^{5}, b^{7}\}$

Para $\langle c \rangle$, los generadores son $\{c^{1}, c^{3}, c^{5}, c^{7},
c^{11}, c^{13}, c^{17}, c^{19}\}$

\Problem{3} Enlista los elementos de los subgrupos $\langle20\rangle$ y
$\langle 10 \rangle$ en $Z_{30}$. Sea $a$ el elemento del grupo de orden 30.
Enlista los elementos de los subgrupos $a^{20}$ y $a^{10}$.

\[
\begin{align*}
  \langle 20 \rangle &= \{0, 10, 20\} \\
  \langle 10 \rangle &= \{0, 10, 20\} \\
\end{align*}
\]

Para $a$

\[
\begin{align*}
  |a| = |\langle a \rangle| &= 30 \\
  mcd(30,20)  &= mcd(30,10) \\
              &= 1 0\\
  |\langle a^{10}\rangle| &= |\langle a^{20}\rangle| \\
                          &= 30 / 10 = 3 \\
  \langle a^{10}\rangle = \langle a^{20} \rangle &= \langle a^{mcd(30, 10)} \rangle \\
                                                 &= \langle a^{mcd(30, 20)}\rangle \\
                                                 &= \langle a^{10}\rangle \\
	\langle a^{10} \rangle                         &= \{e, a^{10}, a^{20}\}
\end{align*}
\]

\Problem{5} Enlista los elementos de los subgrupos $\langle3\rangle$ y
$\langle 7 \rangle$ en $U(20)$

\TheSolution{}

\[U(20) = \{1,3,7,9,11,13,17,19\}\]

\[
\begin{flalign*}
  &3^1 = 3 \\
  &3^2 = 9 \\
  &3^3 = 27 \mod 20 = 7 \\
  &3^4 = 7 \cdot 3 \mod 20 = 1 \\
  &3^5 = 3\cdot1 \\
  &\langle3\rangle = \{3,7,9,1\} \\ \\
  &7^1 = 7 \\
  &7^2 = 49 \mod 20 = 9 \\
  &7^3 = 9\cdot7 \mod 20 = 3 \\
  &7^4 = 3 \cdot 7 \mod 20 = 1 \\
  &\langle7\rangle = \{1,3,7,9\}
\end{flalign*}
\]


\Problem{8} Sea $a$ un elemento de un grupo y sea $|a| = 15$. Computa el orden
de los siguientes elementos de $G$.
\begin{enumerate}[a.]
  \item $a^3, a^6, a^9, a^{12}$
  \item $a^5, a^{10}$
  \item $a^2, a^4, a^8, a^{14}$
\end{enumerate}

\TheSolution{}
\begin{enumerate}[a.]
  \item $5$
  \item $3$
  \item $15$
\end{enumerate}

\Problem{9} ¿Cuantos subgrupos tiene $Z_{20}$? Enlista los generadores de cada
uno de estos subgrupos. Suponga que $G = \langle a \rangle$ y $|a| = 20$.
¿Cuantos de estos subrupos tiene $G$? Enlista los generadores de estos subgrupos.

\TheSolution{}
Usando la propiedad de los grupos, $\langle a\rangle$ es sugrupo de $G$,
entonces contamos con $20$ subgrupos de $G$, donde el generador es obvio. De
igual manera, ya que el orden de $a$ es 20, eso significa que para obtener la
identidad se tuvo que operar y resultar en otros 19 elementos, donde cada uno
de estos es generador de otro subgrupo de $G$. Es decir, que también existen
$20$ subgrupos y cada potencia de $a$ es un generador.

\Problem{10} En $Z_{24}$, enlista todos los generadores para el subgrupo de
orden 8. Sea $G = \langle a \rangle$ y sea $|a| = 24$. Enlista todos los
generadores del subgrupo de orden 8.

\[
\begin{align*}
    Z_{24} &= \{0,1,...,23\} \\
		\langle 24/8 \rangle = \langle 3 \rangle &= {0, 3, 6, 9, 12, 15, 18, 21} \\
		|3| &= |<3>| = 8
\end{align*}
\]
    Generadores para $Z_{24}$ considerando el producto mod 24: $\{3,9,15,21\}$ \\
    Obtenidos a partir de la f\'ormula: $|a|=\frac{n}{mdc(n,k)}$ recorriendo los n\'umeros desde 0 a 24 donde 24 es el orden de $Z_{24}$ \\
    Ahora
    \begin{align*}
    	<a> &= \{e, a, a^{2}, ..., a^{23}\}
    \end{align*}
    Subgrupo de orden 8
    \begin{align*}
    	<a^{\frac{24}{8}}> &= <a^{3}> \\
        <a^{3}> &= \{e, a^{3}, a^{6}, a^{9}, a^{12}, a^{15}, a^{18}, a^{21}\}
    \end{align*}
    Generadores
    \begin{align*}
		<a^{3}> &= \{a^{3}, a^{9}, a^{15}, a^{21}\}
    \end{align*}
    Obtenidos de la misma forma que el inciso anterior
\end{problem}

\Problem{15} Sea $G$ un grupo Abeliano y sea $H = \{g \in G | |g| $ divide a $12
\}$. Demuestra que $H$ es un subgrupo de $G$. ¿Hay algo especial sobre el 12
aqui? ¿Tu demostración sería valida si 12 fuera reemplazada por algun otro
entero positivo? Enuncia el resultado general.

\TheSolution{}
Si $|g|$ es dividido por $12$, entonces $g^{12} = e$. Sean $a$ y $b$ en $H$.
Vemos que ${(ab^-{1})}^{12} = a^{12}{(b^{12})}^{-1} = ee^{-1} = e$, por tanto,
$H$ es subgrupo de $G$.

\Problem{22} Demuestra que un grupo de orden 3 debe ser ciclico.

\TheSolution{}
Sea $G = \{e, a, b\}$. $ab$ debe estar en $G$. $ab$ no puede ser $a$ o $b$
porque implicaria que $a = e$ o $b = e$. Por tanto, $ab = e$ y $b = a^{-1}$,
dejando el grupo como $\{e, a, a^{-1}\}$, un grupo ciclico.

\Problem{28} Por el teorema 4.3, sabemos que $\langle 1000000\rangle$ es el
unico subgrupo de orden 8, asi que sus elementos son los unicos de orden 8:
1000000, 2000000, 3000000, 4000000, 5000000, 6000000, 7000000.

\Problem{36} Demuestra que un grupo finito es la union de subgrupos propios si
y solo si el grupo no es ciclico.

\TheSolution{}
Supongamos que el grupo es la union de grupos propios. Si $G = \langle
a\rangle$, entonces $a$ pertenece a algun subgrupo propio $H_i$ de $G$ y
por tanto $\langle a \rangle = G$ debe ser un subgrupo de $H_i$, lo cual es
una contradicción.

Por otro lado, asumimos que $G$ no es ciclico. Tomamos un elemento $a$
cualquiera de $G$. Como $G$ no es ciclico, $\langle a \rangle$ es un subgrupo
propio de $G$. Podemos repetir este proceso para todos los elementos $a$ y
al unir estos subgrupos propios obtenemos todo $G$.

\Problem{43} Sea $p$ un primo. Si un grupo tiene mas de $p - 1$ elementos de
orden $p$, ¿Por qué el grupo no puede ser ciclico?

\TheSolution{}
Supongamos un grupo ciclico finito. Si tenemos un subgrupo de orden
$p$, este tendra $\phi(p) = p - 1$ elementos de orden $p$ por el teorema 4.4.
Si existiera otro elemento de orden $p$, tendria que existir otro subgrupo de
orden $p$, lo cual contradice la propiedad de los subgrupos finitos
establecida en el teorema 4.3.

\Problem{50} Si G es un grupo Abeliano y contiene subgrupos ciclicos de
ordenes 4 y 6, ¿Qué otros tamaños de subgrupos ciclicos estan contenidos en
$G$? Generalice.

\TheSolution{} Tomamos dos grupos ciclicos $|\langle a \rangle| = 4$ y
$|\langle b\rangle| = 6$. Sabemos que ${(a^4)}^{3}{(b^6)}^2 = ee = e$, asi que
sabemos que 12 divide a $|ab|$. Tambien sabemos que ${(ab)}^{4} = a^4 = e$,
por lo tanto, $|ab| \neq 1, 2$ o $4$. Conversamente, vemos que ${(ab)}^{6} =
a^{6} \neq e$, asi que $|ab| \neq 1, 2, 3, 6$. Por tanto, debe en existir
subgrupos de orden $1, 2, 3, 4, 6$ y $12$.

\Problem{57} Supón que un grupo $G$ tiene al menos nueve elmentos $x$ tales
que $x^8 = e$. ¿Puedes concluir que $G$ no es ciclico? ¿Y si $G$ tiene al
menos cinco elementos $x$ tales que $x^4 = e$? Generalice.

\TheSolution{} Sabemos que $|x|$ divide a $8$. Si $G$ fuese ciclico, entonces
habria exactamente $\phi(8) + \phi(4) + \phi(2) + \phi(1) = 4 + 2 + 1 + 1 = 8$
elementos con ordenes que dividan a $8$ en $G$, asi que $G$ no es ciclico.
Igual en el otro caso, habria $\phi(4) + \phi(2) + \phi(1) = 2 + 1 + 1 = 4$
elementos con ordenes que dividan a 4, por lo que $G$ tampoco seria ciclico.

\Problem{64} Demuestre que $H = \Bigg\{\begin{bmatrix}
  1 & n \\
  0 & 1
\end{bmatrix} | n \in Z\Bigg\}$ es un subgrupo ciclico de $GL(2, R)$

\TheSolution{} Observamos que $\begin{bmatrix}
  1 & n \\
  0 & 1
\end{bmatrix} = \begin{bmatrix}
  1 & 1 \\
  0 & 1
\end{bmatrix}^{n}$, que pertenece a $GL(2, R)$, por lo que H es un subgrupo de
este.

\Problem{71} Demuestra que para cualquier primo $p$ y entero positivo $n$,
$\phi(p^n) = p^n - p^{n - 1}$.

\TheSolution{} Los numeros menores y no relativamente primos a $p^n$ son los
multiplos de $p$: $p, 2p, \ndots p^{n-1}p$. Hay $p^{n-1}$ de estos numeros, y
$p^n$ numeros enteros positivos menores o iguales a $p^n$, llevando el numero
de primos relativos a $p^n$ a $p^n - p^{n-1}$.


\end{document}
