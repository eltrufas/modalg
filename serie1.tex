\documentclass{article}

\usepackage[utf8]{inputenc}
\usepackage{amsmath}
\usepackage{mathtools}
\usepackage{enumerate}
\usepackage{listings}
\usepackage{color}

\definecolor{dkgreen}{rgb}{0,0.6,0}
\definecolor{gray}{rgb}{0.5,0.5,0.5}
\definecolor{mauve}{rgb}{0.58,0,0.82}

\lstset{frame=tb,
  language=Python,
  aboveskip=3mm,
  belowskip=3mm,
  showstringspaces=false,
  columns=flexible,
  basicstyle={\small\ttfamily},
  numbers=none,
  numberstyle=\tiny\color{gray},
  keywordstyle=\color{blue},
  commentstyle=\color{dkgreen},
  stringstyle=\color{mauve},
  breaklines=true,
  breakatwhitespace=true,
  tabsize=3
}
\newcounter{problem}
\newcounter{solution}

\newcommand\Problem[1]{%
  \stepcounter{problem}%
  \textbf{#1.}~%
  \setcounter{solution}{0}%
}

\newcommand\TheSolution{%
  \textbf{Solución:}\\%
}

\newcommand\ASolution{%
  \stepcounter{solution}%
  \textbf{Solución \thesolution:}\\%
}
\setlength{\parindent}{0in}
\setlength{\parskip}{1em}

\title{Primera serie de ejercicios de Álgebra Moderna}
\author{Akiyuki Shinbou}
\date{Mayo 2018}

\begin{document}

\maketitle

\Problem{1} Para $n = 5, 8, 12, 20$ y $25$, encuentra todos los enteros
positivos menores a $n$ y relativamente primos a $n$.

\TheSolution{}
Para $n = 5$: $\{1, 2, 3, 4\}$
Para $n = 8$: $\{1, 3, 5, 7\}$
Para $n = 12$: $\{1, 5, 7, 11\}$
Para $n = 20$: $\{1, 3, 7, 9, 11, 13, 17, 19\}$
Para $n = 25$: $\{1, 2, 3, 4, 6, 7, 8, 9, 11, 12, 13, 14, 16, 17, 18, 19, 21,
22, 23, 24\}$

\Problem{2} Determina $\gcd(2^4 \cdot 3^2 \cdot 5 \cdot  7^2,  2 \cdot 3^3
\cdot 7 \cdot 11)$ y $\lcm(2^3 \cdot 3^2 \cdot 5, 2 \cdot 3^3 \cdot 7 \cdot
11)$

\TheSolution{}
$\gcd(2^4 \cdot 3^2 \cdot 5 \cdot  7^2,  2 \cdot 3^3 \cdot 7 \cdot 11) = 2
\cdot 3^2 \cdot 7$.

$\lcm(2^3 \cdot 3^2 \cdot 5, 2 \cdot 3^3 \cdot 7 \cdot 11) = 2^3 \cdot 3^3
\cdot 5 \cdot 7 \cdot 11$

\Problem{3}
Determine $51 \mod 13$, $342 \mod 85$, $62 \mod 15$, $10 \mod 15$, $(82 \cdot
73) \mod 7$, $(51 + 68) \mod 7$, $(35 \cdot 24) \mod 11$, y $(47 + 68) \mod
11$.

\TheSolution{}
$51 \mod 13 = 12$

$342 \mod 85 = (2 \mod 85 \cdot 171 \mod 85) \mod 85 = (2 \cdot 1) \mod 85 = 2$

$62 \mod 15 = 2$

$10 \mod 15 = 10$

$(82 \cdot 73) \mod 7 = (5 \cdot 3) \mod 7 = 1$

$(51 + 68) \mod 7 = (2 + 5) \mod 7 = 0$

$(35 \cdot 24) \mod 11 = (2 \cdot 2) \mod 11 = 4$

$(47 + 68) \mod 11 = (3 + 2) \mod 11 = 5$

\Problem{4} Encuentra enteros s y t tales que $1 = 7 \cdot s + 11 \cdot t$.
Muestra que $s$ y $t$ no son unicos.

\TheSolution{}
Dos soluciones: $s = 8, t = -5$; $s = 19, t = -12$

\Problem{5} En Florida, el cuarto y quinto digito del final de una licencia de
conducir da el año de nacimiento. Los ultimos tres digitos de un hombre con
mes de nacimiento $m$ y dia de nacimiento $b$ son representados por $40(m -
1)$. Para las mujeres los digitos son $40(m - 1) + b + 500$. Determine las
fechas de nacimiento y sexos correspondientes a los numeros 42218 y 53953.

\TheSolution{}
42218: Hombre; 18 de Junio del 42.

53953: Mujer; 13 de diciembre del 53.

\Problem{7} Para las licencias de conducir en Nueva York previas al septiembre
de 1992, los tres digitos precediendo a los ultimos 3 del numero de un hombre
con mes de nacimiento $m$ y dia de nacimiento $b$ se representan por $63m +
2b$. Para mujeres los digitos son 63m + 2b + 1. Determine las fechas de
nacimiento y los sexos correspondientes a los numeros 248 y 601.

248: Mujer; 29 de Junio.

601: Hombre; 17 de Septiembre

\Problem{7} Demuestra que si $a$ y $b$ son enteros positivos, entonces $ab =
\lcm(a, b)\cdot\gcd(a, b)$.

\TheSolution{} Podemos expresar $a$ y $b$ como factores primos elevados a una
potencia no negativa: $a = p_1^{m_1}\cdot p_2^{m_2}\cdot \ldots\cdot
p_k^{m_k}$ y $b = p_1^{n_1}\cdot p_2^{n_2}\cdot \ldots\cdot p_k^{n_k}$.
Entonces $lcm(a, b) = p_1^{s_1}\cdot p_2^{s_2}\cdot \ldots\cdot p_k^{s_k}$,
donde $s_i = \max(m_i, n_i)$ y $\gcd(a, b) = p_1^{t_1}\cdot p_2^{t_2}\cdot
\ldots\cdot p_k^{t_k}$ donde $t_i = \min(m_i, n_i)$. Ahora, $lcm(a, b)
\cdot \gcd(a, b) = p_1^{m_1 + n_1}\cdot p_2^{m_2 + n_2}\cdot \ldots\cdot
p_k^{m_k + n_k} = ab$

\Problem{8} Supón que $a$ y $b$ son enteros que dividen al entero $c$. Si $a$
y $b$ son primos relativos, demuestra que $ab$ divide a $c$. Muestra, por
ejemplo, que si $a$ y $b$ no son primos relativos, entonces $ab$ no
necesariamente divide a $c$.

\TheSolution{}
Si $a$ y $b$ son primos relativos, entonces los podemos expresar como una
relación lineal $1 = as + bt$. Ademas, como $a$ y $b$ dividen a $c$, sabemos
que existen $u, v \in Z$.



\end{document}
