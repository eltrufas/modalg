\documentclass{article}

\usepackage[utf8]{inputenc}
\usepackage{amsmath}
\usepackage{mathtools}
\usepackage{enumerate}
\usepackage{listings}
\usepackage{color}

\definecolor{dkgreen}{rgb}{0,0.6,0}
\definecolor{gray}{rgb}{0.5,0.5,0.5}
\definecolor{mauve}{rgb}{0.58,0,0.82}

\lstset{frame=tb,
  language=Python,
  aboveskip=3mm,
  belowskip=3mm,
  showstringspaces=false,
  columns=flexible,
  basicstyle={\small\ttfamily},
  numbers=none,
  numberstyle=\tiny\color{gray},
  keywordstyle=\color{blue},
  commentstyle=\color{dkgreen},
  stringstyle=\color{mauve},
  breaklines=true,
  breakatwhitespace=true,
  tabsize=3
}
\newcounter{problem}
\newcounter{solution}

\newcommand\Problem[1]{%
  \stepcounter{problem}%
  \textbf{Problema #1.}~%
  \setcounter{solution}{0}%
}

\newcommand\TheSolution{%
  \textbf{Solución:}\\%
}

\newcommand\ASolution{%
  \stepcounter{solution}%
  \textbf{Solución \thesolution:}\\%
}
\setlength{\parindent}{0in}
\setlength{\parskip}{1em}

\title{Primera serie de ejercicios de Álgebra Moderna}
\author{Akiyuki Shinbou}
\date{Mayo 2018}

\begin{document}

\maketitle

\Problem{1} Para $n = 5, 8, 12, 20$ y $25$, encuentra todos los enteros
positivos menores a $n$ y relativamente primos a $n$.

\TheSolution{}
Para $n = 5$: $\{1, 2, 3, 4\}$

Para $n = 8$: $\{1, 3, 5, 7\}$

Para $n = 12$: $\{1, 5, 7, 11\}$

Para $n = 20$: $\{1, 3, 7, 9, 11, 13, 17, 19\}$

Para $n = 25$: $\{1, 2, 3, 4, 6, 7, 8, 9, 11, 12, 13, 14, 16, 17, 18, 19, 21,
22, 23, 24\}$

\Problem{2} Determina $\gcd(2^4 \cdot 3^2 \cdot 5 \cdot  7^2,  2 \cdot 3^3
\cdot 7 \cdot 11)$ y $lcm(2^3 \cdot 3^2 \cdot 5, 2 \cdot 3^3 \cdot 7 \cdot
11)$

\TheSolution{}
$\gcd(2^4 \cdot 3^2 \cdot 5 \cdot  7^2,  2 \cdot 3^3 \cdot 7 \cdot 11) = 2
\cdot 3^2 \cdot 7$.

$lcm(2^3 \cdot 3^2 \cdot 5, 2 \cdot 3^3 \cdot 7 \cdot 11) = 2^3 \cdot 3^3
\cdot 5 \cdot 7 \cdot 11$

\Problem{3}
Determine $51 \mod 13$, $342 \mod 85$, $62 \mod 15$, $10 \mod 15$, $(82 \cdot
73) \mod 7$, $(51 + 68) \mod 7$, $(35 \cdot 24) \mod 11$, y $(47 + 68) \mod
11$.

\TheSolution{}
$51 \mod 13 = 12$

$342 \mod 85 = (2 \mod 85 \cdot 171 \mod 85) \mod 85 = (2 \cdot 1) \mod 85 = 2$

$62 \mod 15 = 2$

$10 \mod 15 = 10$

$(82 \cdot 73) \mod 7 = (5 \cdot 3) \mod 7 = 1$

$(51 + 68) \mod 7 = (2 + 5) \mod 7 = 0$

$(35 \cdot 24) \mod 11 = (2 \cdot 2) \mod 11 = 4$

$(47 + 68) \mod 11 = (3 + 2) \mod 11 = 5$

\Problem{4} Encuentra enteros s y t tales que $1 = 7 \cdot s + 11 \cdot t$.
Muestra que $s$ y $t$ no son unicos.

\TheSolution{}
Dos soluciones: $s = 8, t = -5$; $s = 19, t = -12$

\Problem{5} En Florida, el cuarto y quinto digito del final de una licencia de
conducir da el año de nacimiento. Los ultimos tres digitos de un hombre con mes
de nacimiento $m$ y dia de nacimiento $b$ son representados por $40(m - 1)$.
Para las mujeres los digitos son $40(m - 1) + b + 500$. Determine las fechas de
nacimiento y sexos correspondientes a los numeros 42218 y 53953.

\TheSolution{}
42218: Hombre; 18 de Junio del 42.

53953: Mujer; 13 de diciembre del 53.

\Problem{6} Para las licencias de conducir en Nueva York previas al septiembre
de 1992, los tres digitos precediendo a los ultimos 3 del numero de un hombre
con mes de nacimiento $m$ y dia de nacimiento $b$ se representan por $63m + 2b$.
Para mujeres los digitos son 63m + 2b + 1. Determine las fechas de nacimiento y
los sexos correspondientes a los numeros 248 y 601.

\TheSolution{}
248: Mujer; 29 de Junio.

601: Hombre; 17 de Septiembre

\Problem{7} Demuestra que si $a$ y $b$ son enteros positivos, entonces $ab =
lcm(a, b)\cdot gcd(a, b)$.

\TheSolution{} Podemos expresar $a$ y $b$ como factores primos elevados a una
potencia no negativa: $a = p_1^{m_1}\cdot p_2^{m_2}\cdot \ldots\cdot p_k^{m_k}$
y $b = p_1^{n_1}\cdot p_2^{n_2}\cdot \ldots\cdot p_k^{n_k}$. Entonces $lcm(a, b) =
p_1^{s_1}\cdot p_2^{s_2}\cdot \ldots\cdot p_k^{s_k}$, donde $s_i = \max(m_i,
n_i)$ y $\gcd(a, b) = p_1^{t_1}\cdot p_2^{t_2}\cdot \ldots\cdot p_k^{t_k}$ donde
$t_i = \min(m_i, n_i)$. Ahora, $lcm(a, b) \cdot \gcd(a, b) = p_1^{m_1 +
n_1}\cdot p_2^{m_2 + n_2}\cdot \ldots\cdot p_k^{m_k + n_k} = ab$.

\Problem{8} Supón que $a$ y $b$ son enteros que dividen al entero $c$. Si $a$ y
$b$ son primos relativos, demuestra que $ab$ divide a $c$. Muestra, por ejemplo,
que si $a$ y $b$ no son primos relativos, entonces $ab$ no necesariamente divide
a $c$.

\TheSolution{} Si $a$ y $b$ son primos relativos, entonces los podemos expresar
como una relación lineal $1 = as + bt$. Ademas, como $a$ y $b$ dividen a $c$,
sabemos que existen $u, v \in Z$ tales que $c = ua$ y $c = vb$. Procede que $c =
cas + cbt = vbas + uabt = ab(vs) + ab(ut) = ab(vs + ut)$, es decir, ab divide a
c.

Para el contrajemplo tomamos $a = 6$, $b = 4$ y $c = 12$. $6$ y $4$ no son
primos relativos y dividen a $12$, sin embargo $6 \cdot 4 = 24$ no divide a
$12$.

\Problem{9} Si $a$ y $b$ son enteros y $n$ es un entero positivo, pruebe que
$a$ mod $n$ = $b$ mod $n$ si y solo si $n$ divide $a-b$

\TheSolution{}
Para probar esto, empecemos por demostrar que:

Si $a$ mod$n=b$ mod$n$ entonces $a-b$ mod$n=0$.

A partir de la hip\'otesis: 

$a$ mod$n-b$ mod$n=0$, y aplicando divisi\'on modular por ambos lados
obtenemos:

$(a$ mod$n-b$ mod$n)$ mod$n=0$ mod$n$

Y haciendo un breve desarrollo de la ecuaci\'on que buscamos encontrar:

$a-b$ mod $n=(a$ mod$n-b$ mod$n)$ mod$n$

Se realiza exactamente lo mismo para demostrar que la proposici\'on es
bicondicional.

\Problem{10} Sean $a$ y $b$ enteros y $d = \gcd(a, b)$. Demuestre que si $a =
da$ y $b = db$, entonces $\gcd(a', b') = 1$.

\TheSolution{}
Sabemos que existen s,t $\in$ $\mathbb{Z}$ tales que $d = as + bt = (da')s +
(db')t = d(sa' + 	tb')$. Dividiendo ambos lados por d tenemos que $1 = sa' +
tb'$. Regresandonos a la definici\'on, si decimos que $d' = \gcd(a', b')$,
entonces $d' | a'$ y $d' | b'$, entonces $d' | tb' + sa'$. Así, $d' | 1$ lo
que significa que $d' = 1$.


\Problem{12} Sean $a$ y $b$ enteros positivos y sea $d = \gcd(a, b)$ y
$m = lcm(a, b)$. Demuestre que si t divide a $a$ y $b$, $t$ divide a $d$.
Demuestre que si $s$ es un multiplo de $a$ y $b$, s es un multiplo de m.

\TheSolution

Suponiendo que $t$ divide a $a$ y $b$. Existe enteros $x,y$ takes que $ax + by
= d$. Porque $t$ divide $a$  y $b$, tambi\'en cualquier combinaci\'on lineal
de $a$ y $b$, por tanto $t$ divide a $d$.

Suponiendo que $s$ es m\'ultiplo de $a$ y $b$. Por el algoritmo de la
divisi\'on, existe un $0\leq r < m$ y $q$ tal que: $S=mq + r$, esto implica
que: $r=s-mq$.

$s$ es un común multiplo de $a$ y $b$, y $m$ es un común múltiplo. Pero
$0 \leq r < m$ y porque $m$ es el mínimo  común múltiplo, entonces
$r=0$. Por lo tanto $s=mq$ 

\Problem{13} Sean $n$ y $a$ enteros positivos y $d = \gcd(a, n)$. Demuestre
que la ecuación $ax \mod n = 1$ tiene una solución si y solo di $d = 1$.

\TheSolution{}

\[
\begin{align}
  ax &= nk + 1, d > 1 \\
  \frac{ax}{d} &= \frac{nk + 1}{d} \\
  \frac{ax}{d} &= \frac{nk}{d} + \frac{1}{d} \\
  a'x &= n'k + \frac{1}{d} 
\end{align}
\]

Esto significa que el modulo de la operacion es $\frac{1}{d}$ y la unica forma
que eso sea 1, es si $d = 1$.

\Problem{15} Demuestra que todo primo mayor a 3 puede ser escrito de la forma
$6n + 1$ o $6n + 5$.

\TheSolution{} Considerando los residuos posibles al dividir un primo entre 6,
observamos que el residuo no puede ser 0, 2 o 4, porque significaria que el
numero es divisible por 2. No puede ser 3, porque implicaria que el numero es
divisible entre 3. Esto nos deja con $6n + 1$ y $6n + 5$ como las formas que
podrian representar un numero primo.

\Problem{16} Determina $7^{1000}$ mod $6$ y $6^{1001}$ mod $7$.

\TheSolution{}
Considerando que $a\cdot b$ mod $n=a$ mod $n\cdot b$ mod $n$, podemos notar
que la primer divisi\'on nos resulta $1$, ya que $7$ mod $6=1$. Mientras que,
por otra parte, $6$ mod $7=6$, por lo que la segunda divisi\'on debemos
resolverta a partir de una observaci\'on adicional. Notemos que $36$ mod $7=1$
y que $6\cdot 36$ mod $7=6$, por la propiedad antes mencionada. Lo que
significa que podemos resumirlo como $6^{n}$ mod $7=1$ si $n$ mod $2=0$, y
$6^{n}$ mod $7=6$ si $n$ mod $2=1$, entonces $6^{1001}$ mod $7=6$.

\Problem{17} Sean $a$, $b$, $s$ y $t$ enteros. Si $a \mod s = b \mod s$,
demuestra que $a \mod s = b \mod s$ y $a \mod t = b \mod t$. ¿Cual es la
condicion de $s$ y $t$ que hacen que lo opuesto sea verdad?

\TheSolution{}
a y b dejan el mismo residuo al ser divididos por st. Es posible entonces
expresar a = st $\cdot$ $q_{1}$ + r y b = st $\cdot$ $q_{2}$ + r. Como a  y b
son divisibles por st y st es divisible por s entonces r es dividido por s por
lo que r = s $\cdot$ $q_{3}$ + $r_{s}$ con 0 $\leq$ $r_{s}$ $<$ s. De esta
forma a = st $\cdot$ $q_{1}$ + s $\cdot$ $q_{3}$ + $r_{s}$ y b = st $\cdot$
$q_{2}$ + s $\cdot$ $q_{3}$ + $r_{s}$. Reordenando:

\begin{enumerate}
  \item $a = s(tq_{1} + q_{3}) + r_{s}$
  \item $b = s(tq_{2} + q_{3}) + r_{s}$
\end{enumerate}

Lo anterior es posible traducirse como que a y b dejan el mismo
residuo al ser divididos por s. Con ello a mod s = b  mod s. 

Un procedimiento análogo se sigue para demostrar que a mod t = b mod t.

La condición necesaria para que se cumpla es que s y t sean primos relativos.


\Problem{19} Demuestre que $\gcd(a, bc) = 1$ si y solo si $\gcd(a, b) = 1$ y
$\gcd(a, c) = 1$.

\TheSolution{}
Si $\gcd(a, bc) = 1$ entonces no existe ningun primo que divida tanto $a$ como
a $bc$. Usando la descomposición en primos y el lema de Euclides, podemos
decir que no existe primo que divida tanto a $a$ como a $b$ o $a$ como a $c$.
Por esto, $\gcd(a, b) = 1$ y $\gcd(a, b) = 1$.

Por otro lado, si $\gcd(a, b) = 1$ y $\gcd(a, b) = 1$, entonces no existe primo
que divida tanto $a$ como $b$ o $a$ como a $c$. Por el lema de euclides tambien
sabemos que no existe primo que divida tanto a $a$ como a $bc$, por lo que
$\gcd(a, bc) = 1$.

\Problem{22} Para cada entero positivo $n$, demuestra que $1 + \cdots + n = n(n
+ 1) / 2$.

\TheSolution{}
Por inducción. El caso base dice que 1 = 1(1 + 1) / 2 = 1.

Para el paso inductivo, asumimos que $1 + \cdots + n = n(n
+ 1) / 2$ y queremos probar que $1 + \cdots + n + n + 1 = (n + 1)((n + 1) + 1)
/ 2$. Sustituyendo la hipotesis obtenemos que $n(n + 1) / 2 + (n + 1) = (n +
1)((n + 1) + 1) / 2$

\[
  \begin{split}
    1 + \cdots + n + (n + 1) &= n(n + 1) / 2 + (n + 1) \\
                             &= (n(n + 1) + 2(n + 1)) / 2 \\
                             &= (n + 2)(n + 1) / 2 \\
                             &= (n + 1)((n + 1) + 1) / 2
  \end{split}
\]

\Problem{23} Para todo entero positivo $n$, pruebe que un conjunto con
exactamente $n$ elementos tiene exactamente $2^{n}$ subconjuntos (contando el
conjunto vacío y el conjunto mismo).

Para esta prueba recurriremos a la inducci\'on matem\'atica. Para el paso base,
cuando $n=1$, tenemos $2$ subgrupos, lo cual es correcto, ya que corresponden
al conjunto vacío y al conjunto mismo.

Para el paso inductivo, vamos a demostrar que un conjunto con $n+1$ elementos
tiene $2^{n+1}$ subgrupos, con la hip\'otesis correspondiente (un conjunto de
$n$ elementos tiene $2^{n}$ subgrupos).

Para ello, es necesario notar que $2^{n+1} = 2\cdot 2^{n}$ Recordemos que por
hipótesis $2^{n}$ es el n\'umero de subgrupos que tiene un conjunto de $n$
elementos. Y si multiplicamos por dos ese número, obtendríamos el total de
subgrupos, ya que son los que ya teníamos, más ahora todos esos subgrupos
iguales incluyendo al nuevo elemento.

\Problem{29} Demuestra que el primer principio de la inducción matematica es
una consecuencia del principio del ordenamiento.

\TheSolution{}
Tomamos $S$ como un conjunto que contiene al elemento $a$ y siempre que $n
\geq a$ pertenece a $S$, entonces $n + 1 \in S$. Debemos demostrar que $S$
contiene a todos los enteros mayores o iguales a $a$. Sean $T$ los enteros
mayores que $a$ que no se encuentran en $T$ y supongamos que $T$ no está vacio.
Sea $b$ el numero mas pequeño en $T$, es decir, que $b - 1 \in S$. Si esto es
así, entonces $(b - 1) + 1 \in S$, lo cual contradice nuestra asumpcion que $b
\nin S$.

\Problem{36} Supón que una un número de identificación de una orden de dinero y
digito de verificación 21720421168 es copiado erroneamente como 27750421168.
¿El digito de verificación detectará el error?

\TheSolution{}
$2172042116 \mod 9 = 8$ y $2775042116 \mod 8 = 9$, así que no, el error no será
detectado.

\Problem{43} El numero de libro estandar internacional de 10 digitos (ISBN-10)
$a_1 a_2 a_3 a_4 a_5 a_6 a_7 a_8 a_9 a_{10}$ tiene la propiedad
$(a_1,a_2,\ldots,a_{10})\cdot(10,9,8,7,6,5,4,3,2,1) \mod 11 = 0$. El digito
$a^{10}$ es el digito de verificación. Verifica el digito para el ISBN-10
asignado a este libro.

\TheSolution{}
El ISBN-10 es 0-547-16509-9. $(0,5,4,7,1,6,5,0,9,9)\cdot(10,9,8,7,6,5,4,3,2,1)
= 209$. $209 \mod 11 = 0$

\Problem{44} Supón que un ISBN-10 tiene una entrada borrada: 0-716?-2841-9.
Determine el digito faltante.

\TheSolution{} La solución a la ecuación $(0,7,1,6,x,2,8,4,1,9) \cdot (10,9,
8,7,6,5,4,3,2,1) \mod 11 = 0$ tal que $0 \leq x \leq 9$ es 7.

\Problem{50} El estado de Utah anexa un noveno digito $a_9$ al numero de ocho
digitos de una licencia de conducir $a_1 a_2 \ldots a_8$ de manera que $(9a_1 +
8a_2 + 7a_3 + 6a_4 + 5a_5 + 4a_6 + 3a_7 + 2a_8 + a_9) \mod 10 = 0$. Si sabes
que el numero de licencia tiene exactamente un digito incorrecto, explica
porque el error no puede estar en la posición 2, 4, 6 o 8.

\TheSolution{}
Cambiar uno de los digitos pares hace que al momento de verificar el resultado
cambie por una cantidad par, sin embargo $(9\cdot1+8\cdot4+7\cdot9+6\cdot1+5
\cdot0+4\cdot5+3\cdot2+2\cdot6+7) \mod 10 = 5$.

\Problem{57} Suponga que $\alpha$, $\beta$ y $\gamma$ son funciones. Si
$\alpha\gamma = \beta\gamma$ y $\gamma$ es uno a uno y sobre, demuestre que
$\alpha = \beta$.

\TheSolution{}
Si $\gamma$ es sobre y uno a uno, entonce existe una función inversa
$\gamma^{-1}$, y multiplicando $\alpha\gamma = \beta\gamma$ por $\gamma^{-1}$
por el lado derecho obtenemos que $\alpha = \beta$.

\end{document}
